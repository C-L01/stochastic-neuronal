\subsection{Jump generator}
Let $\lambda > 0$ be the rate of both the reset jumps and the postsynaptic increments.
The jump-generator $L_{\cJ}^{\lambda, N} : M_b(\potspace^N) \to M_b(\potspace^N)$ is defined as
\begin{equation}
    (L_{\cJ}^{\lambda, N}\varphi) (\bm{x}) \coloneqq \lambda \int_{\potspace^N} \br{\varphi(\bm{y}) - \varphi(\bm{x})} \kappa(\bm{x}, \odif{\bm{y}})
\end{equation}
where kernel $\kappa$ is given by
\begin{equation}
    % \kappa(x,\bm{z},\odif{y}) \coloneqq \indE{[V_F,+\infty)}(x) \delta_{V_R}(\odif{y}) + \indE{(-\infty,V_F)}(x) \delta_{x + \sum_{j=1}^N \omega^{ij} \indE{z^j \geq V_F}}(\odif{y}).
    \kappa(\bm{x},\odif{\bm{y}}) \coloneqq \sum_{i=1}^N \delta_{\bm{\jtarg}^i(\bm{x})}(\odif{\bm{y}}).
\end{equation}
Here
\begin{equation}
    \bm{\jtarg}^i(\bm{x}) \coloneqq (x^1,\dots,x^{i-1},\jtarg^i(\bm{x}),x^{i+1},\dots,x^N)^\top
\end{equation}
where $\jtarg^i(\bm{x})$ denotes the \enquote{jump target} of the potential of the $i$th neuron under state $\bm{x}$, which is given by
\begin{equation}
    \jtarg^i(\bm{x})
    % \coloneqq \indE{[V_F,+\infty)}(x^i) V_R + \indE{(-\infty,V_F)}(x^i) \br{x^i + \sum_{j=1}^N \omega^{ij} \ind{x^j \geq V_F}}
    \coloneqq \begin{cases}
        x^i + \sum_{j=1}^N \omega^{ij} \ind{x^j \geq V_F} &\text{if } x^i < V_F, \\
        V_R &\text{if } x^i \geq V_F.
    \end{cases}
\end{equation}
Note that $\frac{\kappa}{N}$ is a transition function, i.e.,
\begin{itemize}
    \item $\frac{1}{N} \kappa(\bm{x},\cdot) \in \cP(\potspace^N)$ for all $\bm{x} \in \potspace^N$;
    \item $\frac{1}{N} \kappa(\cdot, \Gamma) \in M_b(\potspace^N)$ for all $\Gamma \in \cB(\potspace^N)$.
\end{itemize}

\smallskip

This jump-generator models the two kinds of stochastic jumps:
\begin{align}
    X_{r-}^i &\to V_R &\quad\text{at rate } \lambda \text{ if } X_{r-}^i \geq V_F \\
    X_{r-}^i &\to X_{r-}^i + \sum_{j \in \{j : X_{r-}^j \geq V_F\}} \omega^{ij} &\quad\text{at rate } \lambda \text{ if } X_{r-}^i < V_F.
\end{align}
