\section{Empirical measure}

Instead of tracking the evolution of all individual neurons over time, we are interested in the \textit{empirical measure curve} \(\mu^N \in D_{\cP(\potspace)}[0,T]\).
It is based on the following function,
\begin{equation}
  \Theta_N : D_{\potspace^N}[0,T] \to D_{\cP(\potspace)}[0,T]; \quad (x^1,\dots,x^N) \mapsto \frac{1}{N} \sum_{i=1}^N \delta_{x^i},
\end{equation}
which compresses the information of all neuron position curves to an (empirical) average curve.
The fact that \(\Theta_N\) maps to the space of càdlàg probability measure curves, as claimed, follows directly from the definition of narrow convergence.
To illustrate the concepts we have introduced so far we will nonetheless quickly prove this.
\begin{lemma}\label{lem:range-of-Theta}
  For \(\bm{x} \in D_{\potspace^N}[0,T]\) we have \(\Theta_N(\bm{x}) \in D_{\cP(\potspace)}[0,T]\).
\end{lemma}

\begin{proof}
  First of all, note that \(\Theta_N(\bm{x})_t\) is a convex combination of (Dirac) probability measures for any \(t \in [0,T]\), so since \(\cP(\potspace)\) is convex \(\Theta_N(\bm{x})_t \in \cP(\potspace)\).
  By definition \(\bm{x}\) is right-continuous, meaning for any \(t \in [0,T)\)
  \begin{equation}
    \forall i \in [N] : \lim_{s \downarrow t} x_s^i = x_t^i.
  \end{equation}
  It follows that for all test functions \(\psi \in C_b(\potspace)\),
  \begin{equation}
    \lim_{s \downarrow t} \int_{\potspace^N} \psi \odif{\Theta_N(\bm{x})_s}
    = \lim_{s \downarrow t} \frac{1}{N} \sum_{i=1}^N \psi(x_s^i)
    = \frac{1}{N} \sum_{i=1}^N \psi(x_t^i)
    = \int_{\potspace^N} \psi \odif{\Theta_N(\bm{x})_t}.
  \end{equation}
  By \zcref{def:narrow-conv} this means that \(\Theta_N(\bm{x})_s \wto \Theta_N(\bm{x})_t\) as \(s \downarrow t\), whence \(\Theta_N(\bm{x})\) is right-continuous (with respect to the narrow topology).
  The existence of (narrow) left-limits for \(\Theta_N(\bm{x})\) follows in exactly the same way from their existence for \(\bm{x}\).
  We conclude that \(\Theta_N(\bm{x}) \in D_{\cP(\potspace)}[0,T]\).
\end{proof}

We will now apply \(\Theta_N\) to our curve of neuron potentials to obtain the empirical measure curve.
\begin{definition}\label{def:emp-measure}
  The empirical measure curve of the process \(\bm{X}\) is defined as
  \begin{equation}
    \mu^N \coloneqq \Theta_N(\bm{X}) = \frac{1}{N} \sum_{i=1}^N \delta_{X^i}.
  \end{equation}
  At time \(t \in [0,T]\), the \textit{empirical measure} thus equals
  \begin{equation}
    \mu_t^N = \frac{1}{N} \sum_{i=1}^N \delta_{X_t^i}.
  \end{equation}
  Note that this is a random probability measure on \( \potspace \).
\end{definition}

\begin{corollary}
  The empirical measure curve is càdlàg w.r.t. the narrow topology, i.e.\ \( \mu^N \in D_{\cP(\potspace)}[0,T] \)~a.s.
\end{corollary}
\begin{proof}
  Recall from \zcref{thm:mart-problem-sol} that \(\bm{X} \in D_{\potspace^N}[0,T]\).
  The result now follows from \zcref{lem:range-of-Theta}.
\end{proof}


\subsection{Empirical martingale}\label{subsec:emp-mart}

The compression of information to empirical averages can also be obtained at the martingale level.
This is achieved by restricting the test function argument of \(M\) (from \zcref{thm:mart-problem-sol}) to one of the form
\begin{equation}
  \varphi(\bm{x}) = \br*{ \frac{1}{N} \bigoplus_{i=1}^N  \psi }(\bm{x})
  \coloneqq \frac{1}{N} \sum_{i=1}^N \psi(x^i),
\end{equation}
for some \( \psi \in C_c^1(\potspace) \).
That is, for any such \( \psi \) we define the \textit{empirical martingale} as
\begin{equation}\label{eq:emp-M-derivation-start}
  \begin{split}
    \ol{M}_t^{\lambda, N}[\psi] \coloneqq
    M_t^{\lambda, N}\sbr*{\frac{1}{N} \bigoplus_{i=1}^N \psi}
     & = \frac{1}{N} \sum_{i=1}^N \psi(X_t^i) - \frac{1}{N} \sum_{i=1}^N \psi(X_0^i) - \int_0^t L^{\lambda, N} \br*{\frac{1}{N}\bigoplus_{i=1}^N \psi} (\bm{X}_r) \odif{r} \\
     & = \int_\potspace \psi \odif{\mu_t^N} - \int_\potspace \psi \odif{\mu_0^N}
    - \int_0^t \sum_{i=1}^N b(X_{r}^i) \cdot \frac{1}{N} \partial_{x} \psi(X_{r}^i) \odif{r}                                                                               \\
    % &\quad - \int_0^t \sum_{i=1}^N \int_\potspace \frac{1}{N} \br*{ \psi(y) + \sum_{\substack{j=1 \\ j \neq i}}^N \psi(X_{r}^j) - \sum_{j=1}^N \psi(X_{r}^j) } \lambda \kappa(X_{r}^i, \bm{X}_r, \odif{y}) \odif{r} \\
     & \quad - \int_0^t \int_{\potspace^N} \frac{1}{N} \sum_{i=1}^N \br*{ \psi(y^i) - \psi(X_{r}^i) } \lambda \kappa(\bm{X}_r, \odif{\bm{y}}) \odif{r}                     \\
     & = \int_\potspace \psi \odif{\mu_t^N} - \int_\potspace \psi \odif{\mu_0^N}
    - \int_0^t \int_\potspace b(x) \partial_{x} \psi(x) \mu_r^N(\odif{x}) \odif{r}                                                                              \\
     & \quad - \int_0^t \int_{\potspace^N} \frac{1}{N} \sum_{i=1}^N \br*{ \psi(y^i) - \psi(X_{r}^i) } \lambda \kappa(\bm{X}_r, \odif{\bm{y}}) \odif{r}                     \\
    %
    % &= \frac{1}{N} \sum_{i=1}^N \psi(X_t^i) - \frac{1}{N} \sum_{i=1}^N \psi(X_0^i) \\
    % &\quad - \int_0^t \frac{1}{N} \sum_{i=1}^N \br*{ b(X_{r}^i) \partial_{x} \psi(X_{r}^i) - \int_\potspace \br*{ \psi(y) - \psi(X_{r}^i) } \lambda \kappa(X_{r}^i, \bm{X}_r, \odif{y}) } \odif{r}
  \end{split}
\end{equation}
We now assume \(\omega^{ij} = \frac{\omega}{N}\).
Then, the integrand of the last term can be rewritten as
\begin{equation}\label{eq:kappa-to-mean-kappa}
  \begin{split}
     & \int_{\potspace^N} \frac{1}{N} \sum_{j=1}^N \br*{ \psi(y^j) - \psi(X_{r}^j) } \lambda \kappa(\bm{X}_r, \odif{\bm{y}})                                                                  \\
     & = \frac{\lambda}{N} \sum_{j=1}^N \int_{\potspace^N} \br*{ \psi(y^j) - \psi(X_{r}^j) } \sum_{i=1}^N \delta_{\bm{\jtarg}^i(\bm{X}_r)}(\odif{\bm{y}})                                     \\
     & = \frac{\lambda}{N} \sum_{j=1}^N \sum_{i=1}^N \int_{\potspace^N} \br*{ \psi(y^j) - \psi(X_{r}^j) } \delta_{\jtarg^i(\bm{X}_r)}(\odif{y^i}) \prod_{k \neq i} \delta_{X_r^k}(\odif{y^k}) \\
     & = \frac{\lambda}{N} \sum_{j=1}^N \sum_{i=1}^N \br{ \br*{ \psi(\jtarg^i(\bm{X}_r)) - \psi(X_{r}^j) } \ind{i = j} + (0 - 0) \ind{i \neq j} }                                             \\
     & = \frac{\lambda}{N} \sum_{i=1}^N \br*{ \psi(\jtarg^i(\bm{X}_r)) - \psi(X_{r}^j) }                                                                                                      \\
     & = \frac{\lambda}{N} \sum_{i=1}^N \br*{ \br*{ \psi(V_R) - \psi(X_{r}^i) } \indE{[V_F,+\infty)}(X_{r}^i)
    + \br*{ \psi(X_{r}^i + \frac{\omega}{N} \sum_{j=1}^N \indE{[V_F,+\infty)}(X_{r}^j)) - \psi(X_{r}^i) } \indE{(-\infty,V_F)}(X_{r}^i) }                                                     \\
     & = \lambda \int_\potspace \br*{ \br*{ \psi(V_R) - \psi(x) } \indE{[V_F,+\infty)}(x)
    + \br*{ \psi(x + \omega \mu_r^N([V_F,+\infty))) - \psi(x) } \indE{(-\infty,V_F)}(x) } \mu_{r}^N(\odif{x})                                                           \\
     & = \int_\potspace \int_\potspace (\psi(y) - \psi(x)) \lambda \ol{\kappa}_{\mu_{r}^N}(x, \odif{y}) \mu_{r}^N(\odif{x})
  \end{split}
\end{equation}
for
\begin{equation}
  \ol{\kappa}_\nu(x, \odif{y}) \coloneqq \delta_{\bar{\jtarg}_\nu(x)}(\odif{y}),
  \quad\text{and}\quad
  \ol{\jtarg}_\nu(x) \coloneqq \begin{cases}
    x + \omega \nu([V_F,+\infty)) & \text{if } x < V_F    \\
    V_R                           & \text{if } x \geq V_F
  \end{cases}, \quad \nu \in \cP(\potspace).
\end{equation}
Substituting this into \zcref{eq:emp-M-derivation-start} yields
\begin{equation}\label{eq:emp-M-expression}
  \begin{split}
    \ol{M}_t^{\lambda, N}[\psi]
     & = \int_\potspace \psi \odif{\mu_t^N} - \int_\potspace \psi \odif{\mu_0^N}
    - \int_0^t \int_\potspace \underbrace{\br*{ b(x) \partial_x \psi(x) + \int_\potspace (\psi(y) - \psi(x)) \lambda \ol{\kappa}_{\mu_{r}^N}(x, \odif{y}) }}_{\eqqcolon \ol{L}_{\mu_{r}^N}^{\lambda} [\psi](x)} \mu_{r}^N(\odif{x}) \odif{r} \\
     & = \int_\potspace \psi \odif{\mu_t^N} - \int_\potspace \psi \odif{\mu_0^N}
    - \int_0^t \int_\potspace \ol{L}_{\mu_{r}^N}^{\lambda} [\psi] \odif{\mu_{r}^N} \odif{r}.
  \end{split}
\end{equation}
If we identify test functions with functions\footnote{Not functionals, because as \( \phi \) is not necessarily continuous as a test function, it is also not necessarily (narrowly) continuous as a function on measures.} on measures, i.e.\ \( \phi(\nu) \coloneqq \int_\potspace \phi \odif{\nu} \) for \(\phi \in M_b(\potspace)\), \( \nu \in \cP(\potspace) \), we observe that this expression for \(\ol{M}^{\lambda,N}\) seems very similar to the martingale problem from \zcref{sec:martingale-problem} if we interpret it as a function of \(\mu^N\) instead of \(\bm{X}\).
However, the operator \(\ol{L}_{\mu_{r}^N}^{\lambda}\) has a nonlinear dependence on the empirical measure, so it is not actually of this form.

\begin{lemma}\label{lem:emp-mart-bounded}
  For fixed \( \lambda > 0 \) and \( \psi \in C_c^1(\potspace) \), \( \ol{M}_t^{\lambda, N}[\psi] \) is bounded uniformly in \( N \).
\end{lemma}

\begin{proof}
  This follows directly from the compact support of \( \psi \), the continuity of \( b \) and the fact that the measures involved are probability measures.
  Namely, for all \( r \geq 0 \), \( x \in \potspace \),
  \begin{equation}
    \abs{ \ol{L}_{\mu_{r}^N}^{\lambda} [\psi](x) }
    = \abs{ b(x) \partial_x \psi(x) + \int_\potspace (\psi(y) - \psi(x)) \lambda \ol{\kappa}_{\mu_{r}^N}(x, \odif{y}) }
    \leq \norm{b}_\infty \norm{\partial_x \psi}_\infty + 2 \lambda \norm{\psi}_\infty
    \eqqcolon C_{\psi, \lambda}, \quad x \in \potspace,
  \end{equation}
  which implies for \( t \geq 0 \),
  \begin{equation}
    \abs{ \ol{M}_t^{\lambda, N}[\psi] }
    \overset{\eqref{eq:emp-M-expression}}{\leq} 2 \norm{\psi}_\infty + t C_{\psi, \lambda}.
  \end{equation}
\end{proof}

\subsection{The empirical martingale is a martingale}

The empirical martingale is a martingale, in two ways.
\begin{corollary}\label{cor:easy-emp-M-mart}
  \(\ol{M}_t^{\lambda,N}[\psi]\) is a martingale on \((D_{\potspace^N}[0,T], \mathscr{S}_{\potspace^N}, P^{\lambda, N})\) adapted to the filtration \( \cF = \Set{\cF_t^\bm{X}}_{t\geq0} \).
\end{corollary}
\begin{proof}
  This directly holds by definition of \( \ol{M}_t^{\lambda,N}[\psi] \) in~\zcref{eq:emp-M-derivation-start} and \zcref{thm:mart-problem-sol}.
\end{proof}

However, \(\ol{M}_t^{\lambda,N}[\psi]\) depends on \( \bm{X} \) only through the empirical measure.
The law of the empirical measure curve \(\mu^N = \Theta_N(\bm{X})\) is given by
\begin{equation}
  \lawmu^{\lambda,N} \coloneqq \Theta_N \# P^{\lambda, N} = P^{\lambda, N} \circ \Theta_N^{-1} \in \cP(D_{\cP(\potspace)}[0,T]),
\end{equation}
which is defined on the (pushforward) \( \sigma \)-algebra
\begin{equation}
  \ol{\cF} \coloneqq \Theta_N \# \mathscr{S}_{\potspace^N} \coloneqq \Set{B \subseteq D_{\cP(\potspace)}[0,T] \given \Theta_N^{-1}(B) \in \mathscr{S}_{\potspace^N}}.
\end{equation}
It is easily verified that this indeed defines a \( \sigma \)-algebra.
We will show that the empirical martingale is still a martingale when we interpret it as a random variable on~\((D_{\cP(\potspace)}[0,T], \ol{\cF}, \lawmu^{\lambda,N})\), i.e.\ when we interpret it as a function in \(\mu^N\) instead of \(\bm{X}\).\footnote{We will not distinguish between \( \ol{M}_t^{\lambda,N}[\psi] :  D_{\potspace^N}[0,T] \to \bR \) and \( \ol{M}_t^{\lambda,N}[\psi] :  D_{\cP(\potspace)}[0,T] \to \bR \) notationally, except when necessary.}

\begin{proposition}\label{prop:emp-M-mart}
  \((\ol{M}_t^{\lambda, N}[\psi])_{t\in[0,T]}\) is a martingale on \((D_{\cP(\potspace)}[0,T], \ol{\cF}, \lawmu^{\lambda,N})\) adapted to the filtration \mbox{\( \ol{\cF}_t \coloneqq \Theta_N \# \cF_t^{\bm{X}} \)}.
\end{proposition}
\begin{proof}
  In this proof we will write \( \ol{M}_t \) instead of \( \ol{M}_t^{\lambda, N}[\psi] \) to lighten the notational burden.
  Furthermore, we will have to distinguish between \( \ol{M}_t \) being a function on \( D_{\cP(\potspace)}[0,T] \) or on \( D_{\potspace^N}[0,T] \), which we will do by writing\footnote{As an exception to the previous footnote.}
  \begin{equation}
    \ol{M}_t(\nu) \coloneqq \int_\potspace \psi \odif{\nu_t} - \int_\potspace \psi \odif{\nu_0}
    - \int_0^t \int_\potspace \ol{L}_{\nu_r}^{\lambda} [\psi] \odif{\nu_r} \odif{r},
    \quad \nu \in D_{\cP(\potspace)}[0,T],
  \end{equation}
  for the former, and
  \begin{equation}
    \widehat{\ol{M}}_t(\bm{\omega}) \coloneqq \ol{M}_t(\Theta_N(\bm{\omega})), \quad \bm{\omega} \in D_{\potspace^N}[0,T],
  \end{equation}
  for the latter.

  We will start by proving that \( (\ol{M}_t)_{t\in[0,T]} \) is adapted to \( (\ol{\cF}_t)_{t\in[0,T]} \).
  For an arbitrary \( B \in \cB \), we have to show that
  \mbox{\begin{math}
    \ol{M}_t^{-1} (B) \in \ol{\cF}_t.
  \end{math}}
  We have
  \begin{equation}
    \begin{split}
      \Theta_N^{-1}(\ol{M}_t^{-1} (B))
       & = \Set{ \bm{\omega} \in D_{\potspace^N}[0,T] \given \Theta_N(\bm{\omega}) \in \ol{M}_t^{-1} (B) } \\
       & = \Set{ \bm{\omega} \in D_{\potspace^N}[0,T] \given \ol{M}_t(\Theta_N(\bm{\omega})) \in B }       \\
       & = \Set{ \bm{\omega} \in D_{\potspace^N}[0,T] \given \widehat{\ol{M}}_t(\bm{\omega}) \in B } \\
       & = \widehat{\ol{M}}_t^{-1}(B) \in \cF_t^{\bm{X}}
    \end{split}
  \end{equation}
  where the inclusion holds because \( (\widehat{\ol{M}})_{t\in[0,T]} \) is adapted to \( (\cF_t^{\bm{X}})_{t\in[0,T]} \) by \zcref{cor:easy-emp-M-mart}.
  This implies, by definition of \( \ol{\cF}_t \), that
  \begin{equation}
    \ol{M}_t^{-1} (B) \in \ol{\cF}_t,
  \end{equation}
  proving that \( (\ol{M}_t)_{t\in[0,T]} \) is adapted to \( (\ol{\cF}_t)_{t\in[0,T]} \).

  \medskip

  Integrability holds because for all \( t \in [0,T] \)
  \begin{equation}
    \begin{split}
    \int_{D_{\cP(\potspace)}[0,T]} \abs{\ol{M}_{t}} \odif{\lawmu^{\lambda,N}}
    &= \int_{D_{\cP(\potspace)}[0,T]} \abs{\ol{M}_{t}} \odif{\br{\Theta_N \# P^{\lambda,N}}} \\
    \overset{(*)}&{=} \int_{\Theta_N(D_{\potspace^N}[0,T])} \abs{\ol{M}_{t}} \odif{\br{\Theta_N \# P^{\lambda,N}}} \\
    &= \int_{D_{\potspace^N}[0,T]} \abs{\ol{M}_{t} \circ \Theta_N} \odif{P^{\lambda,N}}
    = \int_{D_{\potspace^N}[0,T]} \abs{\widehat{\ol{M}}_{t}} \odif{P^{\lambda,N}} < \infty,
    \end{split}
  \end{equation}
  since \(\widehat{\ol{M}}_{t}\) is integrable (due to \zcref{cor:easy-emp-M-mart}).
  Note that in step \((*)\) we merely removed a \(\br{\Theta_N \# P^{\lambda,N}}\)-null set.

  \medskip

  Finally, let \(s > t\), \(\Gamma \in \ol{\cF}_t\), then with a similar derivation we get
  \begin{equation}
    \begin{split}
      \int_\Gamma \ol{M}_{s} \odif{\lawmu^{\lambda,N}}
      % &= \int_\Gamma \ol{M}_{s} \odif{\br{\Theta_N \# P^{\lambda,N}}} \\
      &= \int_{\Theta_N(\Theta_N^{-1}(\Gamma))} \ol{M}_{s} \odif{\br{\Theta_N \# P^{\lambda,N}}} \\ % NOTE that Theta_N is not surjective, so the integral domain shrinks here. But it is okay because we are integrating against a measure whose support is a subset of the range of Theta_N. And maybe the fact that Theta_N is injective is important too. Think on this some more, and explain if necessary.
      &= \int_{\Theta_N^{-1}(\Gamma)} \ol{M}_{s} \circ \Theta_N \odif{P^{\lambda,N}} \\
      &= \int_{\Theta_N^{-1}(\Gamma)} \widehat{\ol{M}}_{s} \odif{P^{\lambda,N}} \\
      \overset{(*)}&{=} \int_{\Theta_N^{-1}(\Gamma)} \widehat{\ol{M}}_{t} \odif{P^{\lambda,N}}
      = \int_\Gamma \ol{M}_t \odif{\lawmu^{\lambda,N}},
    \end{split}
  \end{equation}
  where in step \((*)\) we could use the martingale property of \(\widehat{\ol{M}}\) since \(\Gamma \in \ol{\cF}_t\) implies \(\Theta_N^{-1}(\Gamma) \in \cF_t^{\bm{X}}\).
  This means that
  \begin{equation}
    \Exp{\lawmu^{\lambda,N}}{\ol{M}_{s} \given \ol{\cF}_t} = \ol{M}_t.
  \end{equation}
  We conclude that \((\ol{M}_t^{\lambda, N}[\psi])_{t\in[0,T]}\) is a martingale on \((D_{\cP(\potspace)}[0,T], \ol{\cF}, \lawmu^{\lambda,N})\) adapted to the filtration \mbox{\( \ol{\cF}_t \coloneqq \Theta_N \# \cF_t^{\bm{X}} \)}.
\end{proof}

% TODO decide what to do with the following
If the empirical measure \(\mu^N\) would converge a.s. for \( N\to\infty \) w.r.t. the Skorokhod topology to some \( \ol{\mu}^\lambda \in D_{\cP(\potspace)}[0,T] \), it would follow by Proposition 3.5.2 from~\cite{ethierMarkovProcessesCharacterization1985} that \(\mu^N\) would also converge pointwise at the continuity points of \( \ol{\mu}^\lambda \).
% This would make \(\ol{M}_t^{\lambda, N}[\psi]\) converge too.   % TODO would it?
But unfortunately this form of convergence is out of reach.  % TODO why?
Instead, we will focus on obtaining narrow convergence of \(\lawmu^{\lambda,N}\) by means of tightness.
