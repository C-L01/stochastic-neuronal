\section{Empirical measure and empirical martingale}

Instead of tracking the evolution of all individual neurons over time, we are interested in the \textit{empirical measure} curve \(\mu^{\lambda,N} \in D_{\cP(\bR)}[0,T]\).\footnote{We explicitly indicate the dependence on both \(\lambda\) and \(N\), as we will vary them later.}
It is based on the following function,
\begin{equation}
  \Theta_N : D_{\potspace^N}[0,T] \to D_{\cP(\potspace)}[0,T]; \quad (x_\cdot^1,\dots,x_\cdot^N) \mapsto \frac{1}{N} \sum_{i=1}^N \delta_{x_\cdot^i},
\end{equation}
which compresses the information on all neuron positions to an (empirical) average.
\begin{definition}\label{def:emp-measure}
  The empirical measure of the process \(\bm{X}\) at time \(t \in [0,T]\) is defined as
  \begin{equation}
    \mu_t^{\lambda, N} \coloneqq \Theta_N(\bm{X})_t = \frac{1}{N} \sum_{i=1}^N \delta_{X_t^i}.
  \end{equation}
\end{definition}
Note that the measure \(\mu_t^{\lambda, N}\) is a random measure.
As claimed above, it easily follows that the empirical measure curve is càdlàg with respect to the narrow topology.
We will nonetheless prove this to illustrate the concepts we have introduced so far.
\begin{lemma}
  For the empirical measure curve as defined in \zcref{def:emp-measure}, we have \(\mu^{\lambda,N} \in D_{\cP(\bR)}[0,T]\).
\end{lemma}

\begin{proof}
  Recall from \zcref{thm:mart-problem-sol} that \(\bm{X} \in D_{\potspace^N}[0,T]\).
  Thus, \(\bm{X}\) is right-continuous, so for any \(t \in [0,T)\)
  \begin{equation}
    \forall i \in [N] : \lim_{s \downarrow t} X_s^i = X_t^i.
  \end{equation}
  It follows that for any test function \(\psi \in C_b(\potspace)\),
  \begin{equation}
    \lim_{s \downarrow t} \int_{\potspace^N} \psi \odif{\mu_s^{\lambda, N}}
    = \lim_{s \downarrow t} \frac{1}{N} \sum_{i=1}^N \psi(X_s^i)
    = \frac{1}{N} \sum_{i=1}^N \psi(X_t^i)
    = \int_{\potspace^N} \psi \odif{\mu_t^{\lambda, N}}.
  \end{equation}
  By \zcref{def:narrow-conv} this means that \(\mu_s^{\lambda, N} \wto \mu_t^{\lambda, N}\) as \(s \downarrow t\), whence \(\mu^{\lambda,N}\) is right-continuous.
  The existence of left-limits for \(\mu^{\lambda,N}\) follows in exactly the same way from their existence for \(\bm{X}\).
  We conclude that \(\mu^{\lambda,N} \in D_{\cP(\bR)}[0,T]\).
\end{proof}

The compression of information to empirical averages can also be obtained at the martingale level.
This is achieved by restricting the test function argument of \(M\) (from \zcref{thm:mart-problem-sol}) to one of the form
\begin{equation}
  \varphi(\bm{x}) = \br*{ \frac{1}{N} \bigoplus_{i=1}^N  \psi }(\bm{x})
  \coloneqq \frac{1}{N} \sum_{i=1}^N \psi(x^i),
\end{equation}
for some \(\psi \in C_0^1(\potspace)\).
That is, for any such \( \psi \) we define the \textit{empirical martingale} as
\begin{equation}\label{eq:emp-M-derivation-start}
  \begin{split}
    \ol{M}_t^{\lambda, N}[\psi] \coloneqq
    M_t^{\lambda, N}\sbr*{\frac{1}{N} \bigoplus_{i=1}^N \psi}
     & = \frac{1}{N} \sum_{i=1}^N \psi(X_t^i) - \frac{1}{N} \sum_{i=1}^N \psi(X_0^i) - \int_0^t L^{\lambda, N} \br*{\frac{1}{N}\bigoplus_{i=1}^N \psi} (\bm{X}_r) \odif{r} \\
     & = \int_\potspace \psi \odif{\mu_t^{\lambda, N}} - \int_\potspace \psi \odif{\mu_0^{\lambda, N}}
    - \int_0^t \sum_{i=1}^N b(X_{r}^i) \cdot \frac{1}{N} \partial_{x} \psi(X_{r}^i) \odif{r}                                                                               \\
    % &\quad - \int_0^t \sum_{i=1}^N \int_\potspace \frac{1}{N} \br*{ \psi(y) + \sum_{\substack{j=1 \\ j \neq i}}^N \psi(X_{r}^j) - \sum_{j=1}^N \psi(X_{r}^j) } \lambda \kappa(X_{r}^i, \bm{X}_r, \odif{y}) \odif{r} \\
     & \quad - \int_0^t \int_{\potspace^N} \frac{1}{N} \sum_{i=1}^N \br*{ \psi(y^i) - \psi(X_{r}^i) } \lambda \kappa(\bm{X}_r, \odif{\bm{y}}) \odif{r}                     \\
     & = \int_\potspace \psi \odif{\mu_t^{\lambda, N}} - \int_\potspace \psi \odif{\mu_0^{\lambda, N}}
    - \int_0^t \int_\potspace b(x) \partial_{x} \psi(x) \mu_r^{\lambda, N}(\odif{x}) \odif{r}                                                                              \\
     & \quad - \int_0^t \int_{\potspace^N} \frac{1}{N} \sum_{i=1}^N \br*{ \psi(y^i) - \psi(X_{r}^i) } \lambda \kappa(\bm{X}_r, \odif{\bm{y}}) \odif{r}                     \\
    %
    % &= \frac{1}{N} \sum_{i=1}^N \psi(X_t^i) - \frac{1}{N} \sum_{i=1}^N \psi(X_0^i) \\
    % &\quad - \int_0^t \frac{1}{N} \sum_{i=1}^N \br*{ b(X_{r}^i) \partial_{x} \psi(X_{r}^i) - \int_\potspace \br*{ \psi(y) - \psi(X_{r}^i) } \lambda \kappa(X_{r}^i, \bm{X}_r, \odif{y}) } \odif{r}
  \end{split}
\end{equation}
We now assume \(\omega^{ij} = \frac{\omega}{N}\).
Then, the integrand of the last term can be rewritten as
\begin{equation}\label{eq:kappa-to-mean-kappa}
  \begin{split}
     & \int_{\potspace^N} \frac{1}{N} \sum_{j=1}^N \br*{ \psi(y^j) - \psi(X_{r}^j) } \lambda \kappa(\bm{X}_r, \odif{\bm{y}})                                                                  \\
     & = \frac{\lambda}{N} \sum_{j=1}^N \int_{\potspace^N} \br*{ \psi(y^j) - \psi(X_{r}^j) } \sum_{i=1}^N \delta_{\bm{\jtarg}^i(\bm{X}_r)}(\odif{\bm{y}})                                     \\
     & = \frac{\lambda}{N} \sum_{j=1}^N \sum_{i=1}^N \int_{\potspace^N} \br*{ \psi(y^j) - \psi(X_{r}^j) } \delta_{\jtarg^i(\bm{X}_r)}(\odif{y^i}) \prod_{k \neq i} \delta_{X_r^k}(\odif{y^k}) \\
     & = \frac{\lambda}{N} \sum_{j=1}^N \sum_{i=1}^N \br{ \br*{ \psi(\jtarg^i(\bm{X}_r)) - \psi(X_{r}^j) } \ind{i = j} + (0 - 0) \ind{i \neq j} }                                             \\
     & = \frac{\lambda}{N} \sum_{i=1}^N \br*{ \psi(\jtarg^i(\bm{X}_r)) - \psi(X_{r}^j) }                                                                                                      \\
     & = \frac{\lambda}{N} \sum_{i=1}^N \br*{ \br*{ \psi(V_R) - \psi(X_{r}^i) } \indE{[V_F,+\infty)}(X_{r}^i)
    + \br*{ \psi(X_{r}^i + \frac{\omega}{N} \sum_{j=1}^N \indE{[V_F,+\infty)}(X_{r}^j)) - \psi(X_{r}^i) } \indE{(-\infty,V_F)}(X_{r}^i) }                                                     \\
     & = \lambda \int_\potspace \br*{ \br*{ \psi(V_R) - \psi(x) } \indE{[V_F,+\infty)}(x)
    + \br*{ \psi(x + \omega \mu_r^{\lambda, N}([V_F,+\infty))) - \psi(x) } \indE{(-\infty,V_F)}(x) } \mu_{r}^{\lambda, N}(\odif{x})                                                           \\
     & = \int_\potspace \int_\potspace (\psi(y) - \psi(x)) \lambda \ol{\kappa}_r(x, \odif{y}) \mu_{r}^{\lambda, N}(\odif{x})
  \end{split}
\end{equation}
for
\begin{equation}
  \ol{\kappa}_r(x, \odif{y}) \coloneqq \delta_{\bar{\jtarg}_r(x)}(\odif{y}),
  \quad\text{and}\quad
  \ol{\jtarg}_r(x) \coloneqq \begin{cases}
    x + \omega \mu_r^{\lambda, N}([V_F,+\infty)) & \text{if } x < V_F,    \\
    V_R                                          & \text{if } x \geq V_F.
  \end{cases}
\end{equation}
Substituting this into \zcref{eq:emp-M-derivation-start} yields
\begin{equation}
  \begin{split}
    \ol{M}_t^{\lambda, N}[\psi]
     & = \int_\potspace \psi \odif{\mu_t^{\lambda, N}} - \int_\potspace \psi \odif{\mu_0^{\lambda, N}}
    - \int_0^t \int_\potspace \underbrace{\br*{ b(x) \partial_x \psi(x) + \int_\potspace (\psi(y) - \psi(x)) \lambda \ol{\kappa}_r(x, \odif{y}) }}_{\eqqcolon \ol{L}_r^{\lambda, N} [\psi](x)} \mu_{r}^{\lambda, N}(\odif{x}) \odif{r} \\
    & = \int_\potspace \psi \odif{\mu_t^{\lambda, N}} - \int_\potspace \psi \odif{\mu_0^{\lambda, N}}
    - \int_0^t \int_\potspace \ol{L}_r^{\lambda, N} [\psi] \odif{\mu_{r}^{\lambda, N}} \odif{r} \\
  \end{split}
\end{equation}
If we identify test functions with functions on measures, i.e. \( \phi(\nu) \coloneqq \int_\potspace \phi \odif{\nu} \) for \(\phi \in M_b(\potspace)\), \( \nu \in \cP(\potspace) \),\footnote{Because \(\phi\) is not necessarily continuous as a test function, it is also not necessarily (narrowly) continuous as a function on measures.} we observe that this expression for \(\ol{M}^{\lambda,N}\) is very similar to the martingale problem from \zcref{sec:martingale-problem} if we interpret it as a random variable in \(\mu^{\lambda,N}\) instead of \(\bm{X}\).
However, the operator \(\ol{L}_r^{\lambda, N}\) has a nonlinear dependence on the empirical measure through \(\ol{\kappa}\), so it is not actually of this form.

\bigskip

% TODO rewrote all above, so start from here

The law of \((\mu_t^{\lambda, N})_{t\in[0,T]}\) is given by
\begin{equation}
  \lawmu^{\lambda,N} \coloneqq \Theta_N \# P^{\lambda, N} = P^{\lambda, N} \circ \Theta_N^{-1} \in \cP(D_{\cP(\potspace)}[0,T]),
\end{equation}
which is defined on the \( \sigma \)-algebra
\begin{equation}
  \ol{\cF} \coloneqq \Theta_N \# \cF \coloneqq \Set{B \subseteq D_{\cP(\potspace)}[0,T] : \Theta_N^{-1}(B) \in \cF}.
\end{equation}

\begin{corollary}\label{cor:emp-M-mart}
  \(\ol{M}_t^{\lambda, N}[\psi]\) is a martingale on \((D_{\cP(\potspace)}[0,T], \ol{\cF}, \lawmu^{\lambda,N})\) adapted to the filtration \(\ol{\cF}_t \coloneqq \Theta_N \# \cF_t\).
\end{corollary}
\begin{proof}
  Adaptation and integrability are clear from \zcref{thm:mart-problem-sol}. % TODO Are they really? Check for yourself
  For \(s > t\), \(B \in \ol{\cF}_t\),
  \begin{equation}
    \begin{split}
      \int_B \ol{M}_{s}^{\lambda, N}[\psi] \odif{\lawmu^{\lambda,N}}
       & = \int_B \sbr*{ \int_\potspace \psi \odif{\nu_s} - \int_\potspace \psi \odif{\nu_0} - \int_0^s \int_\potspace \ol{L}_r^{\lambda, N} [\psi](x) \nu_{r}(\odif{x}) \odif{r} } \lawmu^{\lambda,N} (\odif{\nu})                                                                          \\
       & = \int_{\Theta_N(\Theta_N^{-1}(B))} \sbr*{ \int_\potspace \psi \odif{\nu_s} - \int_\potspace \psi \odif{\nu_0} - \int_0^s \int_\potspace \ol{L}_r^{\lambda, N} [\psi](x) \nu_{r}(\odif{x}) \odif{r} } \Theta_N \# P^{\lambda, N} (\odif{\nu})                                       \\ % TODO Note that Theta_N is not surjective, so the integral domain shrinks here. But I think it is okay because we are integrating against a measure whose support is a subset of the range of Theta_N. And maybe the fact that Theta_N is injective is important too. Think on this some more, and explain if necessary.
       & = \int_{\Theta_N^{-1}(B)} \sbr*{ \int_\potspace \psi(x) (\Theta_N(\omega))_s(\odif{x}) - \int_\potspace \psi(x) (\Theta_N(\omega))_0(\odif{x}) - \int_0^s \int_\potspace \ol{L}_r^{\lambda, N} [\psi](x) (\Theta_N(\omega))_{r}(\odif{x}) \odif{r} } P^{\lambda, N} (\odif{\omega}) \\
       & = \int_{\Theta_N^{-1}(B)} \sbr*{ \int_\potspace \psi(x) \mu_s^{\lambda,N}(\odif{x}) - \int_\potspace \psi(x) \mu_0^{\lambda,N}(\odif{x}) - \int_0^s \int_\potspace \ol{L}_r^{\lambda, N} [\psi](x) \mu_{r}^{\lambda,N}(\odif{x}) \odif{r} } \odif{P^{\lambda, N}}                   \\
       & = \int_{\Theta_N^{-1}(B)} \ol{M}_s^{\lambda, N}[\psi] \odif{P^{\lambda, N}}                                                                                                                                                                                                         \\
       & = \int_{\Theta_N^{-1}(B)} \ol{M}_t^{\lambda, N}[\psi] \odif{P^{\lambda, N}} % TODO Martingale property, using the definition of contidional expectation and the fact that Theta_N^{-1}(B) \in \cF_t (by definition of \hat{\cF}_t)
      = \int_B \ol{M}_t^{\lambda, N}[\psi] \odif{\lawmu^{\lambda,N}},
    \end{split}
  \end{equation}
  which implies that
  \begin{equation}
    \Exp{\lawmu^{\lambda,N}}{\ol{M}_{s}^{\lambda, N}[\psi] \given \ol{\cF}_t} = \ol{M}_t^{\lambda, N}[\psi].
  \end{equation}
\end{proof}


If \((\mu_t^{\lambda, N})_{t\in[0,T]}\) would converge a.s. for \(N\to\infty\) w.r.t. Skorokhod topology, it would follow by Proposition 5.2 (EK) that it would converge pointwise at continuity points.
This would make \(\ol{M}_t^{\lambda, N}[\psi]\) converge too.
But unfortunately this will not apply; at most \((\lawmu^{\lambda,N})\) will converge narrowly due to being tight.