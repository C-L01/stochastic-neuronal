\section{Preliminaries}

Let $(E,r)$ be a metric space.
We will assume that it is both separable and complete, because we will exclusively work with such spaces.


\subsection{Notation}
\begin{itemize}
    \item $M(E)$ is the space of Borel measurable, real-valued functions on $E$.
    \item $M_b(E) \subset M(E)$ is the space of bounded, Borel measurable, real-valued functions on $E$.
    \item $C_b^{1-}(E) \subset M_b(E)$ is the space of bounded, locally Lipschitz continuous functions.
    \item $\cB(E)$ denotes the Borel $\sigma$-algebra on $E$.
    \item $\cP(E)$ is the space of Borel probability measures on $E$.
\end{itemize}


\subsection{Skorokhod space, topology, metric and \texorpdfstring{$\sigma$}{sigma}-algebra}
The Skorokhod space $D_E[0,\infty)$ is defined as the space of functions $x : [0,\infty) \to E$ which are right-continuous and have existing left-limits.
That is, for $x \in D_E[0,\infty)$ we have 
\begin{equation}
    \forall t \in [0,\infty) : x(t+) \coloneqq \lim_{s \downarrow t} x(s) = x(t), \quad\text{and}\quad x(t-) \coloneqq \lim_{s \uparrow t} x(s)\ \text{exists},
\end{equation}
where the latter holds by convention for $t = 0$.
Often, we will use subscripts to denote evaluation of elements of $D_E[0,\infty)$, i.e., $x_t \coloneqq x(t)$.

We topologize the Skorokhod space $D_E[0,\infty)$ using the (J1) Skorokhod metric as defined in \cite[pp.116-117]{ethier_markov_1985}.
This metric is defined as
\begin{equation}
    d(x,y) \coloneqq \inf_{\lambda \in \Lambda} \cbr*{ \norm{\lambda}_\circ \lor \int_0^\infty e^{-s} d(x,y,\lambda,s) \odif{s} },
\end{equation}
where $\Lambda$ is the set of Lipschitz continuous, strictly increasing bijections that map $[0,\infty)$ onto $[0,\infty)$ for which
\begin{equation}
    \norm{\lambda}_\circ \coloneqq \sup_{s > t \geq 0} \abs*{ \log \frac{\lambda(s)-\lambda(t)}{s-t} } < \infty,
\end{equation}
and
\begin{equation}
    d(x,y,\lambda,s) \coloneqq \sup_{t \geq 0} \cbr{ 1 \land r(x(t \land s), y(\lambda(t) \land s)) }.
\end{equation}
Because $(E, r)$ is both complete and separable, so is $(D_E[0,\infty), d)$ \cite[Theorem 3.5.6]{ethier_markov_1985}.

For an intuitive justification of the machinery in this metric, see \cite{kern_skorokhod_2024}.
Other choices of metric are possible but they induce the same topology, see \cite[pp.166-168]{billingsley_convergence_1999} and \cite[p.122-123]{pollard_convergence_1984}.

\medskip

We define $\mathscr{S}_{E}$ to be the Borel $\sigma$-algebra induced by this Skorokhod topology.
Because $E$ is separable, $\mathscr{S}_{E}$ is generated by the point-evaluation maps \cite[Proposition 3.7.1]{ethier_markov_1985}.


\subsection{Martingale problem on \texorpdfstring{$D_E[0,\infty))$}{D_E[0,infinity)}}\label{sec:martingale-problem}

For our definition of the martingale problem, we will follow Ethier and Kurz \cite[p.174]{ethier_markov_1985}.
Let $(A, D(A)) : M_b(E) \to M_b(E)$ be a linear operator.\footnote{The definition can be extended to multivalued, nonlinear operators, but we will not require that generality.}
We say that a probability measure $P \in \cP(D_E[0,\infty))$ is a solution of the martingale problem for $A$ if for the canonical coordinate process
\begin{equation}
    X_t(\omega) \coloneqq \omega_t, \quad \omega \in D_E[0,\infty), t \geq 0,
\end{equation}
defined on the probability space $(D_E[0,\infty), \mathscr{S}_E, P)$, it holds that for all test functions $f \in D(A)$
\begin{equation}
    t \mapsto f(X_t) - f(X_0) - \int_0^t (Af)(X_s) \odif{s}
\end{equation}
is a martingale with respect to the filtration $\Set{\cF_t^X}_{t\geq0} \coloneqq \Set{\sigma(X_s : s \leq t)}$. % equals \mathscr{S}_E restricted to [0,t], see citation of Prop. 3.7.1 above
This eponymous condition is what makes the problem relevant in the study of stochastic processes; it is satisfied whenever $A$ is the generator of a $X$.

If additionally $(X_0)_\# P = P X_0^{-1} = \mu$ for a prescribed initial distribution $\mu \in \cP(E)$, we say that $P$ is a solution of the martingale problem for $(A,\mu)$.
Such a solution is called unique if for any two solutions the finite-dimensional distributions (of $X$) are identical.
The martingale problem for $(A,\mu)$ is well-posed if (i) a solution exists and (ii) it is unique.
If this holds for all $\mu \in \cP(E)$, we say that the martingale problem for $A$ is well-posed.



\subsection{Tightness in Skorokhod spaces}

\dots
