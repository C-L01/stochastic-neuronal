\subsection{Relative compactness}\label{sec:compactness}

\red{TODO: text about this section}

We start with a small helper lemma.

\begin{lemma}\label{lem:K-set-compact}
  Let \( M > 0 \).
  The set
  \begin{equation}
    \cK \coloneqq \Set{ \mu \in \cP(\bR) \given \int_\bR \abs{x} \mu(\odif{x}) \leq M }
  \end{equation}
  is (narrowly) compact.
\end{lemma}

\begin{proof}
  We begin by noting that the absolute value function has compact sublevel sets.
  \zcref[S]{lem:tightness-char-coercive-function} therefore gives us that \( \cK \) is tight, and thus relatively compact by \zcref{thm:prokhorov}.

  It remains to be shown that \( \cK \) is closed.
  To demonstrate this, suppose \( \Set{\mu_n}_{n\in\bN} \subset \cK \) converges narrowly to some \( \mu \in \cP(\potspace) \).
  It suffices to show that also \( \mu \in \cK \).
  Define for each \( R > 0 \) the function \( \phi_R : \bR \to \bR \) as
  \begin{equation}
    \phi_R(x) \coloneqq \min(\abs{x}, R).
  \end{equation}
  Then \( \phi_R \in C_b(\bR) \) for all \( R > 0 \).
  Hence, the narrow convergence implies
  \begin{equation}\label{eq:mu_n->mu-consequence}
    \lim_{n\to\infty} \int_\bR \phi_R(x) \mu_n(\odif{x}) = \int_\bR \phi_R(x) \mu(\odif{x}).
  \end{equation}
  Since for each \( n \in \bN \) we have \( \mu_n \in \cK \), we have
  \begin{equation}
    \int_\bR \phi_R(x) \mu_n(\odif{x}) \leq \int_\bR \abs{x} \mu_n(\odif{x}) \leq M.
  \end{equation}
  Combined with~\eqref{eq:mu_n->mu-consequence} this yields
  \begin{equation}\label{eq:mu(phiR)<=M}
    \int_\bR \phi_R(x) \mu(\odif{x}) \leq M \quad \forall R > 0.
  \end{equation}

  Now, observe that \( \phi_R \) is monotonically increasing and converges pointwise to the absolute value function for \( R \to \infty \).
  The monotone convergence theorem thus implies
  \begin{equation}
    \int_\bR \abs{x} \mu(\odif{x})
    = \lim_{R\to\infty} \int_\bR \phi_R(x) \mu(\odif{x})
    \overset{\eqref{eq:mu(phiR)<=M}}{\leq} M.
  \end{equation}
  Therefore \( \mu \in \cK \), showing that \( \cK \) is narrowly closed.
\end{proof}

We are now in position to prove the main result.

\begin{theorem}
  For fixed \(\lambda > 0\), \(\Set{ \lawmu^{\lambda, N} }_{N\in\bN} \subset \cP(D_{\cP(\potspace)}[0,T])\) is relatively compact.
  % \begin{equation}
  %     \operatorname{Law}(\mu^N) = \lawmu^{\lambda, N} \wto \bar{\lawmu} \in \mathcal{P}(D_{\mathcal{P}(\potspace)}[0,T])
  % \end{equation}
  % as \(N\to\infty\).
\end{theorem}

\begin{proof}
  By Prokhorov's Theorem (\zcref{thm:prokhorov}), it suffices to show that \(\Set{ \lawmu^{\lambda, N} }_{N\in\bN}\) is tight.
  Because \((\cP(\potspace), \pi)\) is a separable, complete metric space by \zcref{thm:narrow-metrization}, we can apply \zcref{thm:ccc+wmcc=tight}.
  It thus suffices to show that the compact containment condition from \zcref{def:CCC} and the weak modulus of continuity condition from \zcref{def:WMCC} hold for \( \Set{ \lawmu^{\lambda, N} }_{N\in\bN} \).

  \textbf{CCC:}
  To verify the CCC we have to show that
  \begin{equation}
    \forall \eps > 0 : \exists \cK_\eps \Subset \cP(\potspace) :
    \inf_{N\in\bN} \lawmu^{\lambda, N} ( \Set{ \mu \in D_{\cP(\potspace)}[0,T] \given \mu_t \in \cK_\eps\ \forall t \in [0,T] } ) \geq 1 - \eps.
    % \inf_{N\in\bN} P^{\lambda, N} (\mu_t^N \in \cK_\eps,\ 0 \leq t \leq T) \geq 1 - \eps.
  \end{equation}
  Let \(\eps \in (0,1)\), and define
  \begin{equation}
    M_\eps \coloneqq \frac{V_0 + T ( C_\mathrm{df} + \lambda (\abs{V_R} + \abs{\weight}) )}{\eps},
    \quad
    \cK_\eps \coloneqq \Set{ \mu \in \cP(\potspace) \given \int_\potspace \abs{x} \mu(\odif{x}) \leq M_\eps }.
  \end{equation}
  Here \( V_0 \) is the upper bound on the initial potentials from \zcref{assum:X0-bounded}, and \( C_\mathrm{df} \) is the constant from \zcref{assum:b-growth}.
  Observe that because \( M_\eps \geq V_0 \) we have \( \mu_0^N \in \cK_\eps \) a.s., and that \( \cK_\eps \) is compact by \zcref{lem:K-set-compact}.

  Let \( N \in \bN \).
  Recall that \(\lawmu^{\lambda, N} = \Theta_N \sharp P^{\lambda, N}\) is the law of the empirical measure process \(\mu^N\).
  Hence,
  \begin{equation}
    \begin{split}
      \lawmu^{\lambda, N} ( \Set{ \mu \in D_{\cP(\potspace)}[0,T] \given \mu_t \in \cK_\eps\ \forall t \in [0,T] } )
       & = P^{\lambda, N}( \Set{ \bm{x} \in D_{\potspace^N}[0,T] \given (\Theta_N \circ \bm{x})_t \in \cK_\eps\ \forall t \in [0,T] } ) \\
       & = P^{\lambda, N} (\mu_t^N \in \cK_\eps,\ 0 \leq t \leq T).
    \end{split}
  \end{equation}
  It thus suffices to show that
  \begin{equation}
    P^{\lambda, N} (\exists t \in [0,T] : \mu_t^N \notin \cK_\eps) = 1 - P^{\lambda, N} (\mu_t^N \in \cK_\eps,\ 0 \leq t \leq T) \leq \eps.
  \end{equation}

  \medskip

  Define
  \begin{equation}
    \tau \coloneqq  \inf\Set{ t \geq 0 \given \mu_t^N \notin \cK_\eps } \land T
    = \inf\Set{ t \geq 0 \given \int_\potspace \abs{x} \mu_t^N(\odif{x}) > M_\eps } \land T,
  \end{equation}
  where the minimum with \( T \) only serves to set \( \tau = T \) if \( \mu_t^N \in \cK_\eps \) for all \( t \in [0, T] \).
  We do this to ensure that~\( \tau \in [0,T] \).

  Because the (canonical) process \( \bm{X} \) is right-continuous, \( t \mapsto \int_\potspace \abs{x} \mu_t^N(\odif{x}) = \sum_{i=1}^N \abs{X_i^t} \) is as well.
  Since the set~\( (M_\eps, \infty) \) is open, and the filtration is right-continuous (\textbf{assuming} \zcref{cav:right-cont-filtration} is solved), Theorem 2.36 from~\cite{klebanerIntroductionStochasticCalculus2012} implies that the infimum, which is a hitting time, is a stopping time.
  As \( T \) is a constant it is trivially a stopping time, so Theorem 2.37 from~\cite{klebanerIntroductionStochasticCalculus2012} implies that \( \tau \) is a stopping time as well.

  Due to right-continuity, we have
  \begin{equation}
    \exists t \in [0,T] : \mu_t^N \notin \cK_\eps \implies \int_\potspace \abs{x} \mu_\tau^N(\odif{x}) \geq M_\eps,
  \end{equation}
  so Markov's inequality yields
  \begin{equation}\label{eq:ccc:markov-res}
    P^{\lambda, N} (\exists t \in [0,T] : \mu_t^N \notin \cK_\eps)
    \leq P^{\lambda, N} \br*{ \int_\potspace \abs{x} \mu_\tau^N(\odif{x}) \geq M_\eps }
    \leq \frac{1}{M_\eps} \Exp*{P^{\lambda, N}}{\int_\potspace \abs{x} \mu_\tau^N(\odif{x})}
  \end{equation}
  We would like to bound this using \( \ol{M}_\tau^{\lambda, N}[\abs{\cdot}] \).
  However, the absolute value function is not an element of \( C_c^1(\potspace) \).
  To remedy this, we let \( \Set{\psi_n}_{b\in\bN} \subset C_c^1(\potspace)\) be a sequence of nonnegative functions such that \( \psi_n \overset{n\to\infty}{\to} \abs{\cdot} \) pointwise and from below, and \( \partial_x \psi_n(x) \overset{n\to\infty}{\to} \sgn(x) \) for all \( x \in \potspace \).
  % For example that satisfies everything except for the derivative bound (but that should be fixable)
  % \begin{equation}
  %   \psi_n(x) \coloneqq \rho(\tfrac{x}{n}) \cdot \frac{1}{n} \log(\cosh(n x)),    % NOTE: credits to Gemini 2.5 Pro / the asker on https://math.stackexchange.com/questions/1284946/soft-absolute-value
  % \end{equation}
  % where \( \rho \) is a smooth, compactly supported, indicator function of \( [-1,1] \).
  For any \( n \in \bN \) it follows by~\zcref{cor:easy-emp-M-mart} and the optional stopping theorem~\cite[Theorem 7.15]{klebanerIntroductionStochasticCalculus2012} (using \( \tau \leq T \)) that
  \begin{equation}\label{eq:ccc:mart-usage}
    \begin{split}
    \Exp*{P^{\lambda, N}}{\int_\potspace \psi_n \odif{\mu_\tau^N}}
    &= \Exp*{P^{\lambda, N}}{\ol{M}_\tau^{\lambda, N}[\psi_n] + \int_\potspace \psi_n \odif{\mu_0^N} + \int_0^\tau \int_\potspace \ol{L}_{\mu_{s}^N}^{\lambda} [\psi_n] \odif{\mu_{s}^N} \odif{s} } \\
    &= \Exp*{P^{\lambda, N}}{\ol{M}_0^{\lambda, N}[\psi_n]} + \Exp*{P^{\lambda, N}}{\int_\potspace \psi_n \odif{\mu_0^N} + \int_0^\tau \int_\potspace \ol{L}_{\mu_{s}^N}^{\lambda} [\psi_n] \odif{\mu_{s}^N} \odif{s} } \\
    &= \Exp*{P^{\lambda, N}}{\int_\potspace \psi_n \odif{\mu_0^N}} + \Exp*{P^{\lambda, N}}{\int_0^\tau \int_\potspace \ol{L}_{\mu_{s}^N}^{\lambda} [\psi_n] \odif{\mu_{s}^N} \odif{s} }.
    \end{split}
  \end{equation}
  We want to take \( n\to\infty \) on both sides of this identity.
  By the monotone convergence theorem we have
  \begin{equation}
    \lim_{n\to\infty} \Exp*{P^{\lambda, N}}{\int_\potspace \psi_n \odif{\mu_0^N}} = \Exp*{P^{\lambda, N}}{\int_\potspace \abs{x} \mu_0^N(\odif{x})}
    \quad\text{and}\quad
    \lim_{n\to\infty} \Exp*{P^{\lambda, N}}{\int_\potspace \psi_n \odif{\mu_\tau^N}} = \Exp*{P^{\lambda, N}}{\int_\potspace \abs{x} \mu_\tau^N(\odif{x})}.
  \end{equation}
  For the last term we will use the dominated convergence theorem.
  Recalling~\eqref{eq:emp-M-expression}, we have for \( x \in \potspace \), \( s \in [0,\tau] \),
  \begin{equation}\label{eq:ccc:L[psi_n]-bound}
    \begin{split}
    \ol{L}_{\mu_{s}^N}^{\lambda} [\psi_n](x)
    &= b(x) \partial_x \psi_n(x) + \int_\potspace (\psi_n(y) - \psi_n(x)) \lambda \ol{\kappa}_{\mu_{s}^N}(x, \odif{y}) \\
    &= b(x) \partial_x \psi_n(x) + \lambda (\psi_n(\ol{J}_{\mu_{s}^N}(x)) - \psi_n(x)) \\
    \overset{n\to\infty}&{\to} b(x) \sgn(x) + \lambda (\abs{\ol{J}_{\mu_{s}^N}(x)} - \abs{x})
    \end{split}
  \end{equation}
  Using \zcref{assum:b-growth} and~\eqref{eq:ol-kappa-ol-J-def}, we also have
  \begin{equation}
    \begin{split}
    \ol{L}_{\mu_{s}^N}^{\lambda} [\psi_n](x)
    &= b(x) \partial_x \psi_n(x) + \lambda (\psi_n(\ol{J}_{\mu_{s}^N}(x)) - \psi_n(x)) \\
    &\leq C_\mathrm{df} (1 + \abs{x}) \abs{\partial_x \psi_n(x)} + \lambda (\max\Set{ \psi_n(V_R), \psi_n(x + \weight \mu_s^N([V_F, \infty))) } - \psi_n(x)) \\
    % &\leq C_\mathrm{df} (1 + \abs{x}) + \lambda (\psi_n(V_R) + \psi_n(x + \weight \mu_s^N([V_F, \infty))) - \psi_n(x)) \\
    &\leq C_\mathrm{df} (1 + \abs{x}) + \lambda (\abs{V_R} + \abs{x + \weight \mu_s^N([V_F, \infty))}) \\
    &\leq C_\mathrm{df} (1 + \abs{x}) + \lambda (\abs{V_R} + \abs{x} + \abs{\weight \mu_s^N([V_F, \infty))}) \\
    &\leq (C_\mathrm{df} + \lambda) \abs{x} + \lambda (\abs{V_R} + \abs{\weight} + C_\mathrm{df}).
    \end{split}
  \end{equation}
  This upper bound does not depend on \( n \), and due to \( s \leq \tau \) we have
  \begin{equation}
    \begin{split}
    \Exp*{P^{\lambda, N}}{\int_0^\tau \int_\potspace \br{ (C_\mathrm{df} + \lambda) \abs{x} + \lambda (\abs{V_R} + \abs{\weight} + C_\mathrm{df} } \mu_{s}^N(\odif{x}) \odif{s} }
    &= \Exp*{P^{\lambda, N}}{\int_0^\tau \sbr{ (C_\mathrm{df} + \lambda) \int_\potspace \abs{x} \mu_{s}^N(\odif{x}) + \lambda (\abs{V_R} + \abs{\weight}) + C_\mathrm{df} } \odif{s} } \\
    &\leq \Exp*{P^{\lambda, N}}{\int_0^\tau \sbr{ (C_\mathrm{df} + \lambda) M_\eps + \lambda (\abs{V_R} + \abs{\weight}) + C_\mathrm{df} } \odif{s} } \\
    &\leq T \sbr{ (C_\mathrm{df} + \lambda) M_\eps + \lambda (\abs{V_R} + \abs{\weight}) + C_\mathrm{df} }
    \end{split}
  \end{equation}
  which is finite.
  Hence, the dominated convergence theorem yields
  \begin{equation}
    \lim_{n\to\infty} \Exp*{P^{\lambda, N}}{\int_0^\tau \int_\potspace \ol{L}_{\mu_{s}^N}^{\lambda} [\psi_n] \odif{\mu_{s}^N} \odif{s} }
    = \Exp*{P^{\lambda, N}}{\int_0^\tau \int_\potspace \sbr{ b(x) \sgn(x) + \lambda (\abs{\ol{J}_{\mu_{s}^N}(x)} - \abs{x}) } \odif{\mu_{s}^N} \odif{s} }.
  \end{equation}
  Substituting back into \eqref{eq:ccc:mart-usage}, we have
  \begin{equation}
    \begin{split}
    \Exp*{P^{\lambda, N}}{\int_\potspace \abs{x} \mu_\tau^N(\odif{x})} &= \Exp*{P^{\lambda, N}}{\int_\potspace \abs{x} \mu_0^N(\odif{x})} + \Exp*{P^{\lambda, N}}{\int_0^\tau \int_\potspace \sbr{ b(x) \sgn(x) + \lambda (\abs{\ol{J}_{\mu_{s}^N}(x)} - \abs{x}) } \odif{\mu_{s}^N} \odif{s} } \\
    &\leq V_0 + \Exp*{P^{\lambda, N}}{\int_0^\tau \int_\potspace \sbr{ b(x) \sgn(x) + \lambda (\abs{\ol{J}_{\mu_{s}^N}(x)} - \abs{x}) } \odif{\mu_{s}^N} \odif{s} } \\
    \end{split}
  \end{equation}
  where, using \zcref{assum:b-growth} and a derivation similar to~\eqref{eq:ccc:L[psi_n]-bound},
  \begin{equation}
    \begin{split}
    b(x) \sgn(x) + \lambda (\abs{\ol{J}_{\mu_{s}^N}(x)} - \abs{x})
    &\leq C_\mathrm{df} + \lambda (\abs{V_R} + \abs{x} + \abs{\weight} - \abs{x}) \\
    &= C_\mathrm{df} + \lambda (\abs{V_R} + \abs{\weight})
    \end{split}.
  \end{equation}
  Therefore,
  \begin{equation}
    \begin{split}
    \Exp*{P^{\lambda, N}}{\int_\potspace \abs{x} \mu_\tau^N(\odif{x})}
    &\leq V_0 + \Exp*{P^{\lambda, N}}{\int_0^\tau \int_\potspace \sbr{ C_\mathrm{df} + \lambda (\abs{V_R} + \abs{\weight}) } \odif{\mu_{s}^N} \odif{s} } \\
    &\leq V_0 + T ( C_\mathrm{df} + \lambda (\abs{V_R} + \abs{\weight}) ) = \eps M_\eps.
    \end{split}
  \end{equation}
  Thus,
  \begin{equation}
    P^{\lambda, N} (\exists t \in [0,T] : \mu_t^N \notin \cK_\eps) \overset{\eqref{eq:ccc:markov-res}}{\leq}
    \frac{1}{M_\eps} \Exp*{P^{\lambda, N}}{\int_\potspace \abs{x} \mu_\tau^N(\odif{x})}
    \leq \eps,
  \end{equation}
  which implies that \( \Set{\lawmu^{\lambda,N}}_{N \in \bN} \) satisfies the CCC.

  \bigskip

  \textbf{WMCC:}
  To verify that the WMCC holds, we define the following collection of functions on \(\cP(\potspace)\):
  \begin{equation}
    \cG \coloneqq \Set{\nu \mapsto \int_\potspace \psi \odif{\nu} \given \psi \in C_c^1(\potspace)}.
  \end{equation}
  Because we are using the topology of weak convergence on \(\cP(\potspace)\), it follows from \(C_c^1(\potspace) \subset C_b(\potspace)\) that \(\cG \subset C_b(\cP(\potspace))\).
  Furthermore, \( \cG \) is closed under addition, and it separates points as in \zcref{def:s.p.} due to the Riesz representation theorem~\cite[Thm. 2.14]{rudinRealComplexAnalysis2013} and the density of \( C_c^1(\potspace) \) in \( C_c(\potspace) \) (under the uniform norm).
  % TODO Maybe explain this use of Riesz in more detail. It relies on the uniqueness of the measure provided by the theorem (which yields a C_c function on which two distinct measures integrate differently), and an approximation by a C_c^1 function.

  It now suffices to show that \( \Set{g \sharp \lawmu^{\lambda, N}}_{N\in\bN} \subset \cP(D_{\potspace}[0,T])\)\footnote{Still abusing notation to identify \( g : \cP(\potspace) \to \bR \) with the map \( D_{\cP(\potspace)}[0,T] \mapsto D_{\bR}[0,T] \) that applies \( g \) pointwise.} satisfies the AMCC from \zcref{def:MCC} for any \( g \in \cG \).
  Let \( g = g_\psi \in \cG \) and \( \eta > 0 \).
  Set \( C_{\psi, \lambda} \coloneqq \norm{b}_\infty \norm{\partial_x \psi}_\infty + 2\lambda \norm{\psi}_\infty \) as in the proof of \zcref{lem:emp-mart-bounded} (such that \( \ol{L}_{\mu_t^N}^{\lambda,N} [\psi] (x) \leq C_{\psi, \lambda} \) for all \( t \), \( x \)), and define \(\delta \coloneqq \frac{\eta}{4C_{\psi, \lambda}}\).

  Now choose \( N_0 \in \bN \) such that \(\Exp{}{ ( \ol{M}_T^{\lambda, N}[\psi] )^2 } < \frac{\eta^4}{16}\) for \( N \geq N_0 \), which is possible by \zcref{prop:vanishing-variance}, and let \(N \geq N_0\).
  We have to show that
  \begin{equation}
    g \sharp \lawmu^{\lambda, N}(\Set{ x \in D_{\potspace}[0,T] \given w'(x,\delta) \geq \eta }) \leq \eta.
  \end{equation}
  Recall that \(\lawmu^{\lambda, N} = \Theta_N \sharp P^{\lambda, N}\) is the law of the empirical measure process \(\mu^N\).
  Hence,
  \begin{equation}
    \begin{split}
      g \sharp \lawmu^{\lambda, N}(\Set{ x \in D_{\potspace}[0,T] \given w'(x,\delta) \geq \eta })
       & = \lawmu^{\lambda, N}(\Set{ \mu \in D_{\cP(\potspace)}[0,T] \given w'(g \circ \mu,\delta) \geq \eta })              \\
       & = P^{\lambda, N}(\Set{ \bm{x} \in D_{\potspace^N}[0,T] \given w'(g \circ \Theta_N \circ \bm{x},\delta) \geq \eta }) \\
      %  & = P^{\lambda, N}(\Set{ \bm{x} \in D_{\potspace^N}[0,T] \given w'(g \circ \mu^N,\delta) \geq \eta }).
       & = P^{\lambda, N}( w'(g \circ \mu^N,\delta) \geq \eta ).
    \end{split}
  \end{equation}
  Observe that
  \begin{equation}
    g \circ \mu^N = \int_\potspace \psi \odif{\mu^N} = S^{\lambda, N}[\psi] = \ol{M}^{\lambda, N}[\psi] - A^{\lambda, N}[\psi].
  \end{equation}
  Let \( \{t_i\} \) be a \( \delta \)-sparse partition of \( [0,T] \) such that \(\delta < t_i - t_{i-1} < \frac{\eta}{2C_{\psi, \lambda}}\) for all \( i \).
  Recalling \zcref{eq:w-w'-def}, we then obtain
  \begin{equation}
    \begin{split}
      P^{\lambda, N} (w'(g \circ \mu^N, \delta) \geq \eta)
       & \leq P^{\lambda, N}( \max_i w(\ol{M}^{\lambda, N}[\psi], [t_{i-1}, t_i)) + w(A^{\lambda, N}[\psi], [t_{i-1}, t_i) \geq \eta)                                                        \\
       & \leq P^{\lambda, N}( \max_i w(\ol{M}^{\lambda, N}[\psi], [t_{i-1}, t_i)) \geq \frac{\eta}{2}) + P^{\lambda, N}( \max_i w(A^{\lambda, N}[\psi], [t_{i-1}, t_i) \geq \frac{\eta}{2}).
    \end{split}
  \end{equation}
  The first term can be bounded using the bound on the second moment of the empirical martingale.
  Namely, using Doob's inequality~\cite[64]{ethierMarkovProcessesCharacterization1986} followed by Jensen's inequality,
  \begin{equation}
    \begin{split}
      P^{\lambda, N}( \max_i w(\ol{M}^{\lambda, N}[\psi], [t_{i-1}, t_i)) \geq \frac{\eta}{2} )
       & = P^{\lambda, N}( \max_i \sup_{t_{i-1} \leq s < t < t_i} \abs{ \ol{M}_t^{\lambda, N}[\psi] - \ol{M}_s^{\lambda, N}[\psi] } \geq \frac{\eta}{2} ) \\
      % &\leq P^{\lambda, N}( \max_i \sup_{t_{i-1} \leq s < t < t_i} \abs{ \ol{M}_t^{\lambda, N} } + \abs{ \ol{M}_s^{\lambda, N} } \geq \frac{\eta}{2} ) \\
       & \leq P^{\lambda, N}( 2 \sup_{t} \abs{ \ol{M}_t^{\lambda, N}[\psi] } \geq \frac{\eta}{2} )                                                        \\
       & = P^{\lambda, N}( \sup_{t} \abs{ \ol{M}_t^{\lambda, N}[\psi] } \geq \frac{\eta}{4} )                                                             \\
       & \leq \frac{4}{\eta} \Exp{}{\abs{ \ol{M}_T^{\lambda, N}[\psi] }}                                                                                  \\
       & \leq \frac{4}{\eta} \Exp{}{ ( \ol{M}_T^{\lambda, N}[\psi] )^2 }^{1/2}
      % = \frac{4}{\eta} \Exp{}{ [\ol{M}^{\lambda, N}[\psi]]_T }^{1/2}
      < \frac{4}{\eta} \sqrt{\frac{\eta^4}{16}}
      = \eta.
    \end{split}
  \end{equation}
  The second term can be bounded using \( t_i - t_{i-1} < \frac{\eta}{2C_{\psi, \lambda}} \), because \( A^{\lambda, N}[\psi] \) is absolutely continuous.
  In fact, because for any \( t - s < \frac{\eta}{2C_{\psi, \lambda}} \),
  \begin{equation}
    \abs{ A_t^{\lambda, N}[\psi] - A_s^{\lambda, N}[\psi] }
    = \abs*{ \int_s^t \int_{\potspace} \ol{L}_{\mu_r^N}[\psi] \odif{\mu_r^N} \odif{r} }
    \leq C_{\psi,\lambda} (t - s)
    < \frac{\eta}{2}.
  \end{equation}
  we even have
  \begin{equation}
    P^{\lambda, N}( \max_i w(A^{\lambda, N}[\psi], [t_{i-1}, t_i) \geq \frac{\eta}{2}) = 0.
  \end{equation}
  We conclude that
  \begin{equation}
    P^{\lambda, N} (w'(g \circ \mu^N, \delta) \geq \eta)
    < \eta + 0 = \eta,
  \end{equation}
  meaning that \( \Set{g \sharp \lawmu^{\lambda,N}}_{N \in \bN} \) satisfies the AMCC.\
  As \( g \in \cG \) was arbitrary, \( \Set{\lawmu^{\lambda,N}}_{N \in \bN} \) satisfies the WMCC.

  % Since for any \(\psi \in C_b(\potspace)\),
  % \begin{equation}
  %     \sup_{N\in\bN} \Exp{}{ \sup_{t\in[0,T]} \abs{ \int_\potspace \psi \odif{\mu_t^N} } } \leq \norm{\psi}_{\infty},
  % \end{equation}
  % it suffices to show by the Aldous-Rebolledo criterion that both the martingale part and bounded variation part of \(S_t\) have tight laws.
  % For stopping times \(0 \leq S \leq S' \leq S + \delta \leq T\), we have
  % \begin{equation}
  %     \Exp{}{ \ol{M}_{S'}^{\lambda,N} - \ol{M}_{S}^{\lambda,N} }
  %     \overset{\text{(CS)}}{\leq} \Exp{}{ \abs{ \ol{M}_{S'}^{\lambda,N} - \ol{M}_{S}^{\lambda,N} }^2 }^{1/2}
  %     \leq \Exp{}{\sbr{\ol{M}^{\lambda,N}}_{S+\delta} - \sbr{\ol{M}^{\lambda,N}}_{S}}^{1/2}
  %     \overset{N\to\infty}{\to} 0
  % \end{equation}
  % by \zcref{prop:before}.
  % The bounded variation part \(A_{S'} - A_S\) also clearly goes to zero for \(\delta \downarrow 0\).
\end{proof}