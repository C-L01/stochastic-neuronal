\section{Model}

In this section we introduce our neuronal model, which we define as the solution of a martingale problem we will formulate.
The operator generating this martingale problem is based on the dynamics of the model, and consists of a drift part and a jump part.
We will start by treating these parts separately.

\subsection{The drift generator}

Let \( N \in \bN \) be the number of neurons.
Define the drift generator \(L_{\cD}^{N} : M_b(\potspace^N) \to M_b(\potspace^N)\) with \(D(L_{\cD}^N) \coloneqq C_c^1(\potspace^N)\) as
\begin{equation}
  L_{\cD}^{N}\varphi \coloneqq \bm{b} \cdot \nabla_{\bm{x}} \varphi,
\end{equation}
% NOTE domain L-drift: if we didn't assume compact support, I don't think L-drift would map to M_b... You can vanish at infinity with a rapidly varying derivative as long as the second derivative is unbounded https://math.stackexchange.com/questions/947675/derivatives-vanishing-at-infinity.
where \(\bm{b} : \potspace^N \to \bR^N\) is defined as \( \bm{b}(\bm{x}) \coloneqq (b(x^1), \dots, b(x^N))^\top \), for some physical driving force \(b \in C^{1-}(\potspace)\).
% NOTE b conditions: The growth condition that follows this is sufficient for L mapping to M_b, because we assumed the test functions to be compactly supported
\begin{assumption}\label{assum:b-growth}
The driving force \( b \) satisfies
  \begin{equation}
    \forall x \in \potspace : b(x) \sgn(x) \leq C_\mathrm{df},
  \end{equation}
  for some constant \(C_\mathrm{df} > 0\).
  This is equivalent to \( b \) being bounded above on \( (0,\infty) \) and bounded below on \( (-\infty,0) \).
  Furthermore, it satisfies linear growth:
  \begin{equation}
    \abs{b(x)} \leq C_\mathrm{df} (1 + \abs{x}).
  \end{equation}
\end{assumption}
The first assumption implies that the driving force can at most drive the potential away from zero at bounded speed.
A typical example of such a driving force is the time-homogeneous \textit{leaky model}:
\begin{equation}
  b_{\mathrm{leaky}}(x) \coloneqq -(x - V_\mathrm{rest}),
\end{equation}
where \(V_\mathrm{rest} \in (V_R, V_F)\) is the equilibrium potential.

\subsection{Martingale problem of the drift generator}

We will show that the (\(D_{\potspace^N}[0,\infty)\)) martingale problem for \(L_{\cD}^N\) has a solution for any initial condition.
Recall from \zcref{sec:martingale-problem} that we have to show that for each \(\mu_0 \in \cP(\potspace^N)\) there exists a \(P \in \cP(D_{\potspace^N}[0,\infty))\) such that \( \bm{X}_0 \sharp P = \mu_0 \) and, for any test function \(\varphi \in D(L_{\cD}^N)\),
\begin{equation}
  \cD_t^{N}[\varphi]
  \coloneqq \varphi(\bm{X}_t) - \varphi(\bm{X}_0) - \int_0^t L_{\cD}^{N} \varphi(\bm{X}_r) \odif{r}
\end{equation}
is an \(\Set{\cF_t^X}\)-martingale w.r.t. \( P \).
Here \(\bm{X}_t(\bm{\omega}) \coloneqq \bm{\omega}_t\) is the coordinate process on \(D_{\potspace^N}[0,\infty)\).
Due to the form of \(L_{\cD}^N\) we will be able to show this explicitly.

Suppose first \(\mu_0 = \delta_{\bm{x}_0}\) for some \(\bm{x}_0 \in \potspace^N\), and consider the integral initial value problem
\begin{equation}\label{eq:int-IVP}
  \bm{x}_t = \bm{x}_0 + \int_0^t \bm{b}(\bm{x}_r) \odif{r}, \quad t \geq 0.
\end{equation}
Because \(b\) is locally Lipschitz continuous and grows at most linearly by \zcref{assum:b-growth}, a standard ODE result tells us that this equation admits a unique global solution \(\bm{x} = \bm{x}(\bm{x}_0) \in C^1_{\potspace^N}[0,\infty) \subset D_{\potspace^N}[0,\infty)\).

Now define \(P^{\bm{x}_0} \coloneqq \delta_{\bm{x}(\bm{x}_0)} \in \cP(D_{\potspace^N}[0,\infty))\).
Because there is a one-to-one correspondence between \( \bm{x}_0 \) and \( \bm{x}(\bm{x}_0) \), we then have \( \bm{X}_0 \sharp P^{\bm{x}_0} = \delta_{\bm{x}_0} \) (we argue this in more detail for the general case below).
Also, \(\bm{X} = \bm{X}(\bm{\omega}) = \bm{x}\) holds \(P^{\bm{x}_0}\)-a.s.\ which implies that, for any~\(\varphi \in D(L_{\cD}^N)\),
\begin{equation}\label{eq:D=0-a.s.}
  \begin{split}
    \cD_t^{N}[\varphi] & = \varphi(\bm{X}_t) - \varphi(\bm{X}_0) - \int_0^t \bm{b}(\bm{X}_r) \cdot \nabla_{\bm{x}} \varphi(\bm{X}_r) \odif{r} \\
                       & = \varphi(\bm{x}_t) - \varphi(\bm{x}_0) - \int_0^t \dot{\bm{x}}_r \cdot \nabla_{\bm{x}} \varphi(\bm{x}_r) \odif{r}   \\
                       & = \varphi(\bm{x}_t) - \varphi(\bm{x}_0) - \int_0^t \odv{}{r} \varphi(\bm{x}_r) \odif{r}                              \\
                       & = 0 \quad P^{\bm{x}_0}\text{-a.s.}
  \end{split}
\end{equation}
As \(\cD_t^{N}[\varphi]\) is almost surely constant, it is trivially an \(\Set{\cF_t^X}\)-martingale w.r.t. \( P^{\bm{x}_0} \).

Suppose now that \( \mu_0 \in \cP(\potspace^N) \) is arbitrary.
Using the solution for atomic initial conditions we derived above, we define \( P^{\mu_0} \) as the expectation of a conditional probability as follows:
\begin{equation}\label{eq:P-mu-def}
  P^{\mu_0}(B) \coloneqq \Exp{\mu_0}{P^{\bm{x}_0}(B)}
  = \int_{\potspace^N} P^{\bm{z}}(B) \mu_0(\odif{\bm{z}})
  = \int_{\potspace^N} \indE{B}(\bm{x}(\bm{z})) \mu_0(\odif{\bm{z}}),
\end{equation}
for any \( B \in \mathscr{S}_\infty^{\potspace^N} \).

\begin{caveat}
  For~\eqref{eq:P-mu-def} to be well-defined it is necessary for the integrand to be measurable.
  To obtain this, it would be sufficient for \( \bm{x} : \potspace^N \to D_{\potspace^N}[0,\infty) \) to be \( (\cB(\potspace^N), \mathscr{S}_\infty^{\potspace^N}) \)-measurable here.
  When \( \bm{x} : \potspace^N \to D_{\potspace^N}[0,\infty) \) is continuous this holds, but as this continuity is not entirely obvious it should be verified.
  Maybe this can be done using the continuity of the map \( (t, \bm{z}) \mapsto \bm{x}_t(\bm{z}) \), which is a standard ODE result (continuous dependence on initial data).
  If not, it might still hold in this specific case due to the stable nature of \( b \) (\zcref{assum:b-growth}).
  Alternatively, the required measurability of \( \bm{z} \mapsto P^\bm{z}(B) \) could perhaps be established directly using Theorem 4.4.6 from~\cite{ethierMarkovProcessesCharacterization1986}.
\end{caveat}

For any \( B_0 \in \cB(\potspace^N) \), we have
\begin{equation}
  \bm{X}_0 \sharp P^{\mu_0}(B_0) = P^{\mu_0}(\bm{X}_0^{-1} (B_0))
  = \int_{\potspace^N} \indE{\bm{X}_0^{-1} (B_0)}(\bm{x}(\bm{z})) \mu_0(\odif{\bm{z}})
  = \int_{\potspace^N} \indE{B_0}(\bm{z}) \mu_0(\odif{\bm{z}})
  = \mu_0(B_0),
\end{equation}
where we used that \( \bm{X}_0^{-1} (B_0) \) (all càdlàg paths starting in \( B_0 \)) contains \( \bm{x}(\bm{z}) \) (the unique, continuously differentiable solution of~\eqref{eq:int-IVP} starting in \( \bm{z} \)) if and only if \( \bm{z} \in B_0 \).
Therefore, \( \bm{X}_0 \sharp P^{\mu_0} = \mu_0 \).

Finally, for arbitrary~\(\varphi \in D(L_{\cD}^N)\), we have
\begin{equation}
    P^{\mu_0}(\cD_t^{N}[\varphi] = 0)
    = \int_{\potspace^N} P^{\bm{z}}(\cD_t^{N}[\varphi] = 0) \mu_0(\odif{\bm{z}})
    \overset{\eqref{eq:D=0-a.s.}}{=} \int_{\potspace^N} 0 \mu_0(\odif{\bm{z}})
    = 0,
\end{equation}
from which it again follows that \(\cD_t^{N}[\varphi]\) is an \(\Set{\cF_t^X}\)-martingale w.r.t. \( P^{\mu_0} \).

\smallskip

We conclude that the martingale problem for \(L_{\cD}^N\) has a solution for any initial condition.


\subsection{Jump generator}
Let \(\lambda > 0\) be the rate of both the reset jumps and the postsynaptic increments.
The jump-generator \(L_{\cJ}^{\lambda, N} : M_b(\potspace^N) \to M_b(\potspace^N)\) with \( D( L_{\cJ}^{\lambda, N} ) \coloneqq M_b(\potspace^N) \) is defined as
\begin{equation}
  (L_{\cJ}^{\lambda, N}\varphi) (\bm{x}) \coloneqq \lambda \int_{\potspace^N} \br{\varphi(\bm{y}) - \varphi(\bm{x})} \kappa(\bm{x}, \odif{\bm{y}})
\end{equation}
where kernel \( \kappa \) is given by
\begin{equation}
  % \kappa(x,\bm{z},\odif{y}) \coloneqq \indE{[V_F,\infty)}(x) \delta_{V_R}(\odif{y}) + \indE{(-\infty,V_F)}(x) \delta_{x + \sum_{j=1}^N W^{ij} \indE{z^j \geq V_F}}(\odif{y}).
  \kappa(\bm{x},\odif{\bm{y}}) \coloneqq \sum_{i=1}^N \delta_{\bm{\jtarg}^i(\bm{x})}(\odif{\bm{y}}).
\end{equation}
Here
\begin{equation}
  \bm{\jtarg}^i(\bm{x}) \coloneqq (x^1,\dots,x^{i-1},\jtarg^i(\bm{x}),x^{i+1},\dots,x^N)^\top
\end{equation}
where \(\jtarg^i(\bm{x})\) denotes the \enquote{jump target} of the potential of the \(i\)th neuron under state \(\bm{x}\), which is given by
\begin{equation}
  \jtarg^i(\bm{x})
  % \coloneqq \indE{[V_F,\infty)}(x^i) V_R + \indE{(-\infty,V_F)}(x^i) \br{x^i + \sum_{j=1}^N W^{ij} \ind{x^j \geq V_F}}
  \coloneqq \begin{cases}
    x^i + \sum_{j=1}^N W^{ij} \ind{x^j \geq V_F} & \text{if } x^i < V_F,    \\
    V_R                                               & \text{if } x^i \geq V_F.
  \end{cases}
\end{equation}
Note that \(\frac{\kappa}{N}\) is a transition function, i.e.,
\begin{itemize}
  \item \(\frac{1}{N} \kappa(\bm{x},\cdot) \in \cP(\potspace^N)\) for all \(\bm{x} \in \potspace^N\);
  \item \(\frac{1}{N} \kappa(\cdot, \Gamma) \in M_b(\potspace^N)\) for all \(\Gamma \in \cB(\potspace^N)\).
\end{itemize}

\smallskip

This jump generator models the two kinds of stochastic jumps:
\begin{align}
  X_{r-}^i & \to V_R                                                           & \quad\text{at rate } \lambda \text{ if } X_{r-}^i \geq V_F \\
  X_{r-}^i & \to X_{r-}^i + \sum_{j \in \{j : X_{r-}^j \geq V_F\}} W^{ij} & \quad\text{at rate } \lambda \text{ if } X_{r-}^i < V_F.
\end{align}


\subsection{Full generator}

The full generator \(L^{\lambda, N} : M_b(\potspace^N) \to M_b(\potspace^N)\) with \( D(L^{\lambda, N}) \coloneqq C_c^1(\potspace^N) \) is defined as
\begin{equation}
  L^{\lambda, N} \coloneqq L_{\cD}^{N} + L_{\cJ}^{\lambda, N}.
  % L^{\lambda, N} = \ol{L_{\cD}^{N}} + L_{\cJ}^{\lambda, N}.
\end{equation}

% By Theorem 1 from Dempster \cite{dempster_martingale_1998},\footnote{Which follows from a combination of perturbation results from \cite{ethier_markov_1986}.} we have the following result.

\begin{theorem}\label{thm:mart-problem-sol}
  Let \(\lambda > 0\), \( N \in \bN \), and let \(\mu_0 \in \cP(\potspace^N)\) be an initial condition.
  Then the martingale problem for \((L^{\lambda, N}, \mu_0)\) has a solution. % TODO \textit{unique} solution.
  That means that there exists a probability measure \(P^{\lambda, N} \in \cP(D_{\potspace^N}[0,T])\) such that for the canonical coordinate process \(D_{\potspace^N}[0,T] \ni \bm{\omega} \mapsto \bm{X}_t(\bm{\omega}) \coloneqq \bm{\omega}_t\) defined on \((D_{\potspace^N}[0,T], \mathscr{S}_T^{\potspace^N}, P^{\lambda, N})\) we have \( \bm{X}_0 \sharp P^{\lambda, N} = \mu_0 \), and for each \(\varphi \in D(L^{\lambda, N}) = C_c^1(\potspace^N)\),
  \begin{equation}\label{eq:M-def}
    M_t^{\lambda, N}[\varphi]
    \coloneqq \varphi(\bm{X}_t) - \varphi(\bm{X}_0) - \int_0^t L^{\lambda, N} \varphi(\bm{X}_r) \odif{r}
  \end{equation}
  is a martingale w.r.t.\ the natural filtration \(\Set{\cF_t}_{t\geq0} \coloneqq \Set{\cF_t^\bm{X}}_{t\geq0}\).
\end{theorem}

\begin{proof}
  This follows from Proposition 4.10.2 in~\cite{ethierMarkovProcessesCharacterization1986}, except for \zcref{cav:MP-sol-on-[0,inf)}.
\end{proof}

\begin{caveat}\label{cav:MP-sol-on-[0,inf)}
  Proposition 4.10.2, like all results in~\cite{ethierMarkovProcessesCharacterization1986}, applies to \( D_E[0,\infty) \) instead of \( D_E[0,T] \).
  In stating \zcref{thm:mart-problem-sol} as we did, we are assuming that Proposition 4.10.2 can be adapted to \( D_E[0,T] \).
  Intuitively this should not be a problem, just restricts the domain, but it is technically more involved since the \( \sigma \)-algebra \( \mathscr{S}_T^{\potspace^N} \) on which we want \( P^{\lambda, N} \) to be defined is not obviously the one we get when restricting the elements of the sets in \( \mathscr{S}_\infty^{\potspace^N} \) (on which the \( P^{\lambda, N} \) provided by Proposition 4.10.2 is defined).
  Especially since the relation between the topologies is unclear per \zcref{cav:D_T-D_inf-equiv}.
\end{caveat}

\begin{remark}\label{rem:uniqueness}
  To obtain well-posedness we would also need to establish uniqueness for the solution from \zcref{thm:mart-problem-sol}.
  This could possibly be achieved using Theorem 4.4.1 or Corollary 4.4.4 from~\cite{ethierMarkovProcessesCharacterization1986}, as stated on the bottom of page 255, using that \( L^{\lambda, N} \) is dissipative (which follows easily from Lemma 4.2.1); or Theorem 4.4.6 combined with Theorem 4.10.3 using well-posedness of the martingale problem for \( L_{\cD}^{\lambda, N} \).
  It might be necessary to enlarge \( D(L_{\cD}^N) \) and / or tighten the assumptions on \( b \) to get any of these tools to work.
\end{remark}

\begin{caveat}\label{cav:right-cont-filtration}
  We would like for the filtration \( \Set{\cF_t}_{t\geq0} \) to satisfy the usual conditions.
  In particular, we will require it to be right-continuous in \zcref{sec:compactness}.
  It is however unclear if this is the case.
  If \( (\bm{X}, \Set{\cF_t}_{t\geq0}, P^{\lambda,N}) \) were a Markov process this would likely follow from~\cite[Theorem 2.3.4]{chungLecturesMarkovProcesses1982} provided all null sets are included, but we also do not know if \( (\bm{X}, \Set{\cF_t}_{t\geq0}, P^{\lambda,N}) \) is a Markov process (although this would follow as a bonus if~\cite[Theorem 4.4.1]{ethierMarkovProcessesCharacterization1986} was used for \zcref{rem:uniqueness}).
  % NOTE: The natural filtration of a right-continuous process is not necessarily right-continuous. See https://math.stackexchange.com/questions/3316024/mathcal-f-t-mathcal-f-t-iff-x-t-is-right-continuous?rq=1.
\end{caveat}

For simplicity, we will assume that the initial distribution \( \mu_0 \) of the potentials is concentrated on a bounded subset of \( \potspace^N \).
This could for instance be \( [V_R, V_F]^N \).

\begin{assumption}\label{assum:X0-bounded}
  There exists a constant \( V_0 > 0 \) such that for all \( j \in N \) we have \( X_0^j \leq V_0 \) a.s.
\end{assumption}

% We introduce the following SDE
% \begin{equation}
%     \bm{X}_t = \bm{X}_0 + \int_0^t \bm{b}(\bm{X}_{r-}) \odif{r} + \sum
% \end{equation}

% \dots

% Following Theorem 5.13 from \cite{klebanerIntroductionStochasticCalculus2012}, we can equate the weak solutions of this SDE to the solutions of the \textit{martingale problem}.

% \dots

% If we assume \(W^{ij} = \weight / N\), and for \(\varphi = \frac{1}{N} \bigoplus_i \psi\), the non-spike jump part simplifies to
% \begin{equation}
%     \lambda \int \partial_x \psi (x) \weight \mu_r^\lambda([V_F, \infty)) \mu_r(\odif{x})
% \end{equation}
