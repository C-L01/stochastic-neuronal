\section{Model}

In this section we introduce our neuronal model, which we define as the solution of a martingale problem we will formulate.
The operator generating this martingale problem is based on the dynamics of the model, and consists of a drift part and a jump part.
We will start by treating these parts separately.

\subsection{The drift generator}

Let \( N \in \bN \) be the number of neurons.
Define the drift generator \(L_{\cD}^{N} : M_b(\potspace^N) \to M_b(\potspace^N)\) with \(D(L_{\cD}^N) \coloneqq C_c^1(\potspace^N)\) as
\begin{equation}
  L_{\cD}^{N}\varphi \coloneqq \bm{b} \cdot \nabla_{\bm{x}} \varphi,
\end{equation}
% TODO domain L-drift: I don't think that L-drift maps to M_b... You can vanish at infinity with a rapidly varying derivative as long as the second derivative is unbounded https://math.stackexchange.com/questions/947675/derivatives-vanishing-at-infinity. So assume compact support I guess? (Or do that for b...)
where \(\bm{b} : \potspace^N \to \bR^N\) is defined as \( \bm{b}(\bm{x}) \coloneqq (b(x^1), \dots, b(x^N))^\top \), where \(b \in C^{1-}(\potspace)\) is the physical driving force.
% NOTE b conditions: The growth condition that follows this is sufficient for L mapping to M_b, because we assumed the test functions to be compactly supported
We will assume that \(b\) satisfies the following linear growth assumption:
\begin{equation}
  \forall x \in \potspace : \abs{b(x)} \leq C (1 + \abs{x}),
\end{equation}
for some constant \(C > 0\).
\newline
A typical example of such a driving force is the time-homogeneous \textit{leaky model}:
\begin{equation}
  b_{\mathrm{leaky}}(x) \coloneqq -(x - V_\mathrm{rest}),
\end{equation}
where \(V_\mathrm{rest} \in (V_R, V_F)\) is the equilibrium potential.

\subsection{Martingale problem of the drift generator}

We will show that the \(D_{\potspace^N}[0,\infty)\) martingale problem for \(L_{\cD}^N\) always has a solution.
Recall from \zcref{sec:martingale-problem} that we have to show that for each \(\mu \in \cP(\potspace^N)\) there exists a unique \(P \in \cP(D_{\potspace^N}[0,\infty))\) such that \( (\bm{X}_0)_\# P = \mu \) and for any test function \(\varphi \in D(L_{\cD}^N)\),
\begin{equation}
  \cD_t^{N}[\varphi]
  \coloneqq \varphi(\bm{X}_t) - \varphi(\bm{X}_0) - \int_0^t L_{\cD}^{N} \varphi(\bm{X}_r) \odif{r}
\end{equation}
is an \(\Set{\cF_t^X}\)-martingale.
Here \(\bm{X}_t(\bm{\omega}) \coloneqq \bm{\omega}_t\) is the coordinate process on \(D_{\potspace^N}[0,\infty)\).

Due to the form of \(L_{\cD}^N\), we will be able to do this directly.
Suppose first \(\mu = \delta_{\bm{x}^0}\) for some \(\bm{x}^0 \in \potspace^N\), and consider the integral initial value problem
\begin{equation}
  \bm{x}_t = \bm{x}^0 + \int_0^t \bm{b}(\bm{x}_r) \odif{r}, \quad t \geq 0.
\end{equation}
Because \(b\) is locally Lipschitz continuous and satisfies the linear growth condition, this equation admits a unique global solution \(\bm{x} \in C^1_{\potspace^N}[0,\infty) \subset D_{\potspace^N}[0,\infty)\)~\cite{ode_result}.

Now choose \(P \coloneqq \delta_{\bm{x}} \in \cP(D_{\potspace^N}[0,\infty))\).
By construction we then have \(\bm{X} = \bm{X}(\bm{\omega}) = \bm{x}\) \(P\)-a.s.
Therefore, for any~\(\varphi \in D(L_{\cD}^N)\),
\begin{equation}
  \begin{split}
    \cD_t^{N}[\varphi] & = \varphi(\bm{X}_t) - \varphi(\bm{X}_0) - \int_0^t \bm{b}(\bm{X}_r) \cdot \nabla_{\bm{x}} \varphi(\bm{X}_r) \odif{r} \\
                       & = \varphi(\bm{x}_t) - \varphi(\bm{x}^0) - \int_0^t \dot{\bm{x}}_r \cdot \nabla_{\bm{x}} \varphi(\bm{X}_r) \odif{r}   \\
                       & = \varphi(\bm{x}_t) - \varphi(\bm{x}^0) - \int_0^t \odv{}{r} \varphi(\bm{x}_r) \odif{r}                              \\
                       & = 0 \quad P\text{-a.s.}
  \end{split}
\end{equation}
As \(\cD_t^{N}[\varphi]\) is almost surely constant, it is trivially an \(\Set{\cF_t^X}\)-martingale.


\subsection{Jump generator}
Let \(\lambda > 0\) be the rate of both the reset jumps and the postsynaptic increments.
The jump-generator \(L_{\cJ}^{\lambda, N} : M_b(\potspace^N) \to M_b(\potspace^N)\) with \( D( L_{\cJ}^{\lambda, N} ) \coloneqq M_b(\potspace^N) \) is defined as
\begin{equation}
  (L_{\cJ}^{\lambda, N}\varphi) (\bm{x}) \coloneqq \lambda \int_{\potspace^N} \br{\varphi(\bm{y}) - \varphi(\bm{x})} \kappa(\bm{x}, \odif{\bm{y}})
\end{equation}
where kernel \( \kappa \) is given by
\begin{equation}
  % \kappa(x,\bm{z},\odif{y}) \coloneqq \indE{[V_F,+\infty)}(x) \delta_{V_R}(\odif{y}) + \indE{(-\infty,V_F)}(x) \delta_{x + \sum_{j=1}^N \omega^{ij} \indE{z^j \geq V_F}}(\odif{y}).
  \kappa(\bm{x},\odif{\bm{y}}) \coloneqq \sum_{i=1}^N \delta_{\bm{\jtarg}^i(\bm{x})}(\odif{\bm{y}}).
\end{equation}
Here
\begin{equation}
  \bm{\jtarg}^i(\bm{x}) \coloneqq (x^1,\dots,x^{i-1},\jtarg^i(\bm{x}),x^{i+1},\dots,x^N)^\top
\end{equation}
where \(\jtarg^i(\bm{x})\) denotes the \enquote{jump target} of the potential of the \(i\)th neuron under state \(\bm{x}\), which is given by
\begin{equation}
  \jtarg^i(\bm{x})
  % \coloneqq \indE{[V_F,+\infty)}(x^i) V_R + \indE{(-\infty,V_F)}(x^i) \br{x^i + \sum_{j=1}^N \omega^{ij} \ind{x^j \geq V_F}}
  \coloneqq \begin{cases}
    x^i + \sum_{j=1}^N \omega^{ij} \ind{x^j \geq V_F} & \text{if } x^i < V_F,    \\
    V_R                                               & \text{if } x^i \geq V_F.
  \end{cases}
\end{equation}
Note that \(\frac{\kappa}{N}\) is a transition function, i.e.,
\begin{itemize}
  \item \(\frac{1}{N} \kappa(\bm{x},\cdot) \in \cP(\potspace^N)\) for all \(\bm{x} \in \potspace^N\);
  \item \(\frac{1}{N} \kappa(\cdot, \Gamma) \in M_b(\potspace^N)\) for all \(\Gamma \in \cB(\potspace^N)\).
\end{itemize}

\smallskip

This jump generator models the two kinds of stochastic jumps:
\begin{align}
  X_{r-}^i & \to V_R                                                           & \quad\text{at rate } \lambda \text{ if } X_{r-}^i \geq V_F \\
  X_{r-}^i & \to X_{r-}^i + \sum_{j \in \{j : X_{r-}^j \geq V_F\}} \omega^{ij} & \quad\text{at rate } \lambda \text{ if } X_{r-}^i < V_F.
\end{align}


\subsection{Full generator}

The full generator \(L^{\lambda, N} : M_b(\potspace^N) \to M_b(\potspace^N)\) of our model is defined as
\begin{equation}
  L^{\lambda, N} = \ol{L_{\cD}^{N}} + L_{\cJ}^{\lambda, N}.
\end{equation}
Here \(\ol{L_{\cD}^{N}}\) is the (weak) closure of \(L_{\cD}^N\). % TODO closure: Figure out if this is true / if I want to use and argue this. The idea would be to use EK Theorem 4.4.1 and/or Corollary 4.4.4 as suggested on bottom of p.255

% By Theorem 1 from Dempster \cite{dempster_martingale_1998},\footnote{Which follows from a combination of perturbation results from \cite{ethier_markov_1985}.} we have the following result.

\begin{theorem}\label{thm:mart-problem-sol}
  Let \(\lambda > 0\), \( N \in \bN \), and let \(\nu \in \cP(\potspace^N)\) be an initial condition.
  Then the martingale problem for \((L^{\lambda, N}, \nu)\) has a solution.% TODO \textit{unique} solution.
  That means that there exists a probability measure \(P^{\lambda, N} \in \cP(D_{\potspace^N}[0,T])\) such that for the stochastic process \(D_{\potspace^N}[0,T] \ni \bm{\omega} \mapsto \bm{X}_t(\bm{\omega}) \coloneqq \bm{\omega}_t\) defined on \((D_{\potspace^N}[0,T], \mathscr{S}_{\potspace^N}, P^{\lambda, N})\) and each \(\varphi \in C_c^1(\potspace^N)\), % TODO this should be the domain of L, so if I keep (/ work out how to use) the closure of L_drift it might be larger
  \begin{equation}\label{eq:M-def}
    M_t^{\lambda, N}[\varphi]
    \coloneqq \varphi(\bm{X}_t) - \varphi(\bm{X}_0) - \int_0^t L^{\lambda, N} \varphi(\bm{X}_r) \odif{r}
  \end{equation}
  is a martingale w.r.t.\ the natural filtration \(\Set{\cF_t}_{t\geq0} \coloneqq \Set{\cF_t^\bm{X}}_{t\geq0}\).    % NOTE: this is the natural filtation only because of X being right-continuous, per E&K p.173
\end{theorem}

\begin{proof}
  This follows from Proposition 4.10.2 in~\cite{ethierMarkovProcessesCharacterization1985}.
\end{proof}


% {\Large Insert MP justification / equivalent SDE}

% We introduce the following SDE
% \begin{equation}
%     \bm{X}_t = \bm{X}_0 + \int_0^t \bm{b}(\bm{X}_{r-}) \odif{r} + \sum
% \end{equation}

% \dots

% Following Theorem 5.13 from \cite{klebanerIntroductionStochasticCalculus2012}, we can equate the weak solutions of this SDE to the solutions of the \textit{martingale problem}.

% \dots

% If we assume \(\omega^{ij} = \omega / N\), and for \(\varphi = \frac{1}{N} \bigoplus_i \psi\), the non-spike jump part simplifies to
% \begin{equation}
%     \lambda \int \partial_x \psi (x) \omega \mu_r^\lambda([V_F, \infty)) \mu_r(\odif{x})
% \end{equation}
