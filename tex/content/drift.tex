\section{Model}

\subsection{The drift generator}

Let \(N \in \bN\) be the number of neurons.
Define the drift generator \(L_{\cD}^{N} : M_b(\potspace^N) \to M_b(\potspace^N)\) with \(D(L_{\cD}^N) \coloneqq C^1(\potspace^N) \cap C_0(\potspace^N)\) as
\begin{equation}
  L_{\cD}^{N}\varphi \coloneqq \bm{b} \cdot \nabla_{\bm{x}} \varphi,
\end{equation}
where \(\bm{b} : \potspace^N \to \bR^N\) is defined as \(\bm{b}(\bm{x}) \coloneqq (b(x^1), \dots, b(x^N))^\top\), where \(b \in C_b^{1-}(\potspace)\) is the physical driving force.
We will assume that \(b\) satisfies the following linear growth assumption:
\begin{equation}
  \forall x \in \potspace : \abs{b(x)} \leq C (1 + \abs{x}),
\end{equation}
for some constant \(C > 0\).
\newline
A typical example of such a driving force is the time-homogeneous \textit{leaky model}:
\begin{equation}
  b_{\mathrm{leaky}}(x) \coloneqq -(x - V_\mathrm{rest}),
\end{equation}
where \(V_\mathrm{rest} \in (V_R, V_F)\) is the equilibrium potential.

\subsection{Martingale problem of the drift generator}

We will show that the \(D_{\potspace^N}[0,\infty)\) martingale problem for \(L_{\cD}^N\) always has a solution.
Recall from \zcref{sec:martingale-problem} that we have to show that for each \(\mu \in \cP(\potspace^N)\) there exists a unique \(P \in \cP(D_{\potspace^N}[0,\infty))\) such that \((\bm{X}_0)_\# P = \mu\) and for any test function \(\varphi \in D(L_{\cD}^N)\),
\begin{equation}
  \cD_t^{N}[\varphi]
  \coloneqq \varphi(\bm{X}_t) - \varphi(\bm{X}_0) - \int_0^t L_{\cD}^{N} \varphi(\bm{X}_r) \odif{r}
\end{equation}
is an \(\Set{\cF_t^X}\)-martingale.
Here \(\bm{X}_t(\bm{\omega}) \coloneqq \bm{\omega}_t\) is the coordinate process on \(D_{\potspace^N}[0,\infty)\).

Due to the form of \(L_{\cD}^N\), we will be able to do this directly.
Suppose first \(\mu = \delta_{\bm{x}^0}\) for some \(\bm{x}^0 \in \potspace^N\), and consider the integral initial value problem
\begin{equation}
  \bm{x}_t = \bm{x}^0 + \int_0^t \bm{b}(\bm{x}_r) \odif{r}, \quad t \geq 0.
\end{equation}
Because \(b\) is locally Lipschitz continuous and satisfies the linear growth condition, this equation admits a unique global solution \(\bm{x} \in C^1_{\potspace^N}[0,\infty) \subset D_{\potspace^N}[0,\infty)\) \cite{ode_result}.

Now choose \(P \coloneqq \delta_{\bm{x}} \in \cP(D_{\potspace^N}[0,\infty))\).
By construction we then have \(\bm{X} = \bm{X}(\bm{\omega}) = \bm{x}\) \(P\)-a.s.
Therefore, for any \(\varphi \in D(L_{\cD}^N)\),
\begin{equation}
  \begin{split}
    \cD_t^{N}[\varphi] & = \varphi(\bm{X}_t) - \varphi(\bm{X}_0) - \int_0^t \bm{b}(\bm{X}_r) \cdot \nabla_{\bm{x}} \varphi(\bm{X}_r) \odif{r} \\
    & = \varphi(\bm{x}_t) - \varphi(\bm{x}^0) - \int_0^t \dot{\bm{x}}_r \cdot \nabla_{\bm{x}} \varphi(\bm{X}_r) \odif{r}   \\
    & = \varphi(\bm{x}_t) - \varphi(\bm{x}^0) - \int_0^t \odv{}{r} \varphi(\bm{x}_r) \odif{r}                              \\
    & = 0 \quad P\text{-a.s.}
  \end{split}
\end{equation}
As \(\cD_t^{N}[\varphi]\) is almost surely constant, it is trivially an \(\Set{\cF_t^X}\)-martingale.
% Uniqueness holds because...
